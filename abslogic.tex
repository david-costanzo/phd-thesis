\section{The Abstract Logic}
\label{abslogic}

The previous section demonstrated how we can impose behavior preservation in
the context of Separation Logic, without making Separation Logic any more
difficult to use or any less powerful. We next need to show how behavior
preservation can provide benefits over the standard, weaker notion of locality.
In order to do this, it will help to have a formal, abstract view of
behavior-preserving Separation Logic. This section will describe how 
our strong locality fits into a context similar to that of 
Abstract Separation Logic~\cite{coy07}. With a minor amount of work, the 
logic of Section~\ref{concrete} can be molded into a particular instance 
of the abstract logic presented here.

We define a \emph{separation algebra} to be a set of states $\Sigma$, along with a partial associative
and commutative operator $\dt : \Sigma \to \Sigma \rightharpoonup \Sigma$. The disjointness relation $\sigma_0 \# \sigma_1$ holds iff 
$\sigma_0 \dt \sigma_1$ is defined, and the substate relation $\sigma_0 \preceq \sigma_1$ holds iff there is some
$\sigma_0'$ such that $\sigma_0 \dt \sigma_0' = \sigma_1$. A particular element of $\Sigma$ is 
designated as a unit state, denoted $u$, with the property that for any $\sigma$, $\sigma \# u$ and $\sigma \dt u = \sigma$. We require
the $\dt$ operator to be cancellative, meaning that $\sigma \dt \sigma_0 = \sigma \dt \sigma_1 \Rightarrow \sigma_0 = \sigma_1$.

An \emph{action} is a set of pairs of type $\Sigma \cup \{\fault,\dvg\} \times \Sigma \cup \{\fault,\dvg\}$.
We require the following
two properties: (1) actions always relate $\fault$ to $\fault$ and $\dvg$ to $\dvg$, and never relate $\fault$ or $\dvg$ to 
anything else; and (2) actions are total, in the sense that for any $\tau$, there exists some $\tau'$ such that $\relatebr{A}{\tau}{\tau'}$
(recall from Section~\ref{locality} that we use $\tau$ to range over elements of $\Sigma \cup \{\fault,\dvg\}$). 
Note that these two requirements are preserved over the standard composition of relations, as well as over both finitary and infinite unions.
We write $\id$ to represent the identity action $\{(\tau,\tau) \pipe \tau \in \Sigma \cup \{\fault,\dvg\}\}$.

Note that it is more standard in the literature to have the domain of actions
range only over $\Sigma$~--- we use $\Sigma \cup \{\fault,\dvg\}$ here because it has the pleasant effect of making $\den{C_1;C_2}$
correspond precisely to standard composition. Intuitively, once an execution goes wrong, it continues to go wrong, and once an execution
diverges, it continues to diverge.

A \emph{local action} is an action $A$ that satisfies the following four properties, which respectively correspond to Safety Monotonicity,
Termination Equivalence, the Forwards Frame Property, and the Backwards Frame Property from Section~\ref{locality}:
\begin{align*}
& \text{1.) } \lnot \relatebr{A}{\sigma_0}{\fault} \land \sigma_0 \# \sigma_1 \Longrightarrow 
  \lnot \relatebr{A}{(\sigma_0 \dt \sigma_1)}{\fault} \\
& \text{2.) } \lnot \relatebr{A}{\sigma_0}{\fault} \land \sigma_0 \# \sigma_1 \Longrightarrow 
  (\relatebr{A}{\sigma_0}{\dvg} \iff \relatebr{A}{(\sigma_0 \dt \sigma_1)}{\dvg}) \\
& \text{3.) } \relatebr{A}{\sigma_0}{\sigma_0'} \land \sigma_0 \# \sigma_1 \Longrightarrow 
  \sigma_0' \# \sigma_1 \land \relatebr{A}{(\sigma_0 \dt \sigma_1)}{(\sigma_0' \dt \sigma_1)} \\
& \text{4.) } \lnot \relatebr{A}{\sigma_0}{\fault} \land \relatebr{A}{(\sigma_0 \dt \sigma_1)}{\sigma'} \Longrightarrow
\exists \sigma_0' \such \sigma' = \sigma_0' \dt \sigma_1 \land \relatebr{A}{\sigma_0}{\sigma_0'}
\end{align*}

We denote the set of all local actions by \locact.
We now show that the set of local actions is closed under composition and (possibly infinite) union. We use the notation
$A_1;A_2$ to denote composition, and $\bigcup \A$ to denote union (where $\A$ is a possibly infinite set of actions). The formal
definitions of these operations follow. Note that we require that $\A$ be non-empty. This is necessary because $\bigcup \emp$
is $\emp$, which is not a valid action. Unless otherwise stated, whenever we write $\bigcup \A$, there will always be an implicit
assumption that $\A \neq \emp$.
\begin{align*} 
\relatebr{A_1;A_2}{\tau}{\tau'} & \iff \exists \tau'' \such \relatebr{A_1}{\tau}{\tau''} \land \relatebr{A_2}{\tau''}{\tau'} \\
\relatebr{\bigcup \A}{\tau}{\tau'} & \iff \exists A \in \A \such \relatebr{A}{\tau}{\tau'} \quad \quad (\A \neq \emp)
\end{align*}

\begin{lem}
If $A_1$ and $A_2$ are local actions, then $A_1;A_2$ is a local action.
\end{lem}
\ifextended
\begin{proof}
It will be useful to first note that
$\relatebr{A_1;A_2}{\sigma}{\fault}$ iff either
$\relatebr{A_1}{\sigma}{\fault}$ or there exists some $\sigma'$ such
that $\relatebr{A_1}{\sigma}{\sigma'}$ and
$\relatebr{A_2}{\sigma'}{\fault}$. This is due to the fact that we know
$\relatebr{A_2}{\fault}{\fault}$ and $\lnot
\relatebr{A_2}{\dvg}{\fault}$. Similarly, it is also the case that
$\relatebr{A_1;A_2}{\sigma}{\dvg}$ iff either
$\relatebr{A_1}{\sigma}{\dvg}$ or there exists some $\sigma'$ such that
$\relatebr{A_1}{\sigma}{\sigma'}$ and $\relatebr{A_2}{\sigma'}{\dvg}$.

For Safety Monotonicity, suppose that $\sigma_0 \# \sigma_1$ 
and $\lnot \relatebr{A_1;A_2}{\sigma_0}{\fault}$. 
Suppose by way of
contradiction that $\relatebr{A_1;A_2}{(\sigma_0 \dt \sigma_1)}{\fault}$. Since $\lnot
\relatebr{A_1;A_2}{\sigma_0}{\fault}$ and
$\relatebr{A_2}{\fault}{\fault}$, we have $\lnot
\relatebr{A_1}{\sigma_0}{\fault}$. Thus by Safety Monotonicity of $A_1$,
$\lnot \relatebr{A_1}{(\sigma_0 \dt \sigma_1)}{\fault}$. By our note
above, we see that there must be some $\sigma$ such that
$\relatebr{A_1}{(\sigma_0 \dt \sigma_1)}{\sigma}$ and
$\relatebr{A_2}{\sigma}{\fault}$. By the Backwards Frame Property of
$A_1$, there must be a $\sigma_0'$ such that $\sigma = \sigma_0' \dt
\sigma_1$ and $\relatebr{A_1}{\sigma_0}{\sigma_0'}$.  Thus we have that
$\relatebr{A_2}{(\sigma_0' \dt \sigma_1)}{\fault}$, and so Safety
Monotonicity of $A_2$ tells us that
$\relatebr{A_2}{\sigma_0'}{\fault}$. Hence
$\relatebr{A_1;A_2}{\sigma_0}{\fault}$, which is a contradiction.

For Termination Equivalence, suppose that $\sigma_0 \# \sigma_1$ 
and $\lnot \relatebr{A_1;A_2}{\sigma_0}{\fault}$. Then
we also have $\lnot \relatebr{A_1}{\sigma_0}{\fault}$, 
since we have $\relatebr{A_2}{\fault}{\fault}$. 

For the forward direction, suppose that $\relatebr{A_1;A_2}{\sigma_0}{\dvg}$. 
By the note above, there are two possible situations. In the first situation, we have $\relatebr{A_1}{\sigma_0}{\dvg}$. By
Termination Equivalence of $A_1$, this implies that $\relatebr{A_1}{(\sigma_0 \dt \sigma_1)}{\dvg}$, and so
$\relatebr{A_1;A_2}{(\sigma_0 \dt \sigma_1)}{\dvg}$, as desired. In the second situation, there is a state $\sigma$
such that $\relatebr{A_1}{\sigma_0}{\sigma}$ and $\relatebr{A_2}{\sigma}{\dvg}$. By the Forwards Frame Property of
$A_1$, we see that $\sigma \# \sigma_1$ and $\relatebr{A_1}{(\sigma_0 \dt \sigma_1)}{(\sigma \dt \sigma_1)}$. Now note
that we must have $\lnot \relatebr{A_2}{\sigma}{\fault}$, because otherwise we would be able to derive 
$\relatebr{A_1;A_2}{\sigma_0}{\fault}$, which is a contradiction. Therefore, by Termination Equivalence of $A_2$,
we have $\relatebr{A_2}{(\sigma \dt \sigma_1)}{\dvg}$. Hence we get $\relatebr{A_1;A_2}{(\sigma_0 \dt \sigma_1)}{\dvg}$,
as desired.

For the backward direction, suppose that $\relatebr{A_1;A_2}{(\sigma_0 \dt \sigma_1)}{\dvg}$.  
Again by the note above, there are two
possible situations. In the first situation, we have
$\relatebr{A_1}{(\sigma_0 \dt \sigma_1)}{\dvg}$. By Termination
Equivalence of $A_1$, this implies that
$\relatebr{A_1}{\sigma_0}{\dvg}$, and so
$\relatebr{A_1;A_2}{\sigma_0}{\dvg}$, as desired. In the second
situation, there is a state $\sigma$ such that $\relatebr{A_1}{(\sigma_0 \dt \sigma_1)}{\sigma}$ 
and $\relatebr{A_2}{\sigma}{\dvg}$. By the
Backwards Frame Property of $A_1$, there must be a $\sigma_0'$ such
that $\sigma = \sigma_0' \dt \sigma_1$ and
$\relatebr{A_1}{\sigma_0}{\sigma_0'}$. Now note that we must have $\lnot
\relatebr{A_2}{\sigma_0'}{\fault}$, because otherwise we would be able
to derive $\relatebr{A_1;A_2}{\sigma_0}{\fault}$, which is a
contradiction. Therefore, by Termination Equivalence of $A_2$, we have
$\relatebr{A_2}{\sigma_0'}{\dvg}$. Hence we get
$\relatebr{A_1;A_2}{\sigma_0}{\dvg}$, as desired.

For the Forwards Frame Property, suppose that $\sigma_0 \# \sigma_1$
and $\relatebr{A_1;A_2}{\sigma_0}{\sigma_0'}$.  Then there exists a
$\tau$ such that $\relatebr{A_1}{\sigma_0}{\tau}$ and
$\relatebr{A_2}{\tau}{\sigma_0'}$.  Furthermore, $\tau$ cannot be
$\fault$ or $\dvg$ since $\relatebr{A_2}{\tau}{\sigma_0'}$~--- thus let
$\tau$ be $\sigma_0''$. By the Forwards Frame Property of $A_1$, we
have $\sigma_0'' \# \sigma_1$ and $\relatebr{A_1}{(\sigma_0 \dt
  \sigma_1)}{(\sigma_0'' \dt \sigma_1)}$. Therefore, by the Forwards
Frame Property of $A_2$, we have $\sigma_0' \# \sigma_1$ and
$\relatebr{A_2}{(\sigma_0'' \dt \sigma_1)}{(\sigma_0' \dt \sigma_1)}$.
Hence $\sigma_0' \# \sigma_1$ and $\relatebr{A_1;A_2}{(\sigma_0 \dt
  \sigma_1)}{(\sigma_0' \dt \sigma_1)}$, as desired.

For the Backwards Frame Property, suppose that $\lnot
\relatebr{A_1;A_2}{\sigma_0}{\fault}$ and $\relatebr{A_1;A_2}{(\sigma_0
  \dt \sigma_1)}{\sigma'}$. Then, repeating some reasoning from
earlier in this proof, we have $\lnot \relatebr{A_1}{\sigma_0}{\fault}$,
and there exists a $\sigma$ such that $\relatebr{A_1}{(\sigma_0 \dt
  \sigma_1)}{\sigma}$ and $\relatebr{A_2}{\sigma}{\sigma'}$. By the
Backwards Frame Property of $A_1$, we get $\sigma = \sigma_0' \dt
\sigma_1$ and $\relatebr{A_1}{\sigma_0}{\sigma_0'}$. Now note that
$\lnot \relatebr{A_2}{\sigma_0'}{\fault}$, because otherwise we would be
able to derive $\relatebr{A_1;A_2}{\sigma_0}{\fault}$, which is a
contradiction. Therefore, by the Backwards Frame Property of $A_2$, we
get $\sigma' = \sigma_0'' \dt \sigma_1$ and
$\relatebr{A_2}{\sigma_0'}{\sigma_0''}$. Hence $\sigma' = \sigma_0'' \dt
\sigma_1$ and $\relatebr{A_1;A_2}{\sigma_0}{\sigma_0''}$, as desired.
\end{proof}
\fi

\begin{lem}
If every $A$ in the set $\A$ is a local action, then $\bigcup \A$ is a local action.
\end{lem}

\ifextended
\begin{proof}
For Safety Monotonicity, suppose $\sigma_0 \# \sigma_1$ and 
$\lnot \relatebr{\bigcup \A}{\sigma_0}{\fault}$. Suppose by way
of contradiction that $\relatebr{\bigcup \A}{(\sigma_0 \dt \sigma_1)}{\fault}$. Then there is some $A \in \A$ such that
$\relatebr{A}{(\sigma_0 \dt \sigma_1)}{\fault}$. By Safety Monotonicity of $A$, we get $\relatebr{A}{\sigma_0}{\fault}$.
But this means that $\relatebr{\bigcup \A}{\sigma_0}{\fault}$, which is a contradiction.

For Termination Equivalence, suppose that $\sigma_0 \# \sigma_1$ and $\lnot \relatebr{\bigcup \A}{\sigma_0}{\fault}$.
This means that for every $A \in \A$, $\lnot \relatebr{A}{\sigma_0}{\fault}$. For the forward direction, suppose that 
$\relatebr{\bigcup \A}{\sigma_0}{\dvg}$. Then $\relatebr{A}{\sigma_0}{\dvg}$ for some $A \in \A$. Thus 
Termination Equivalence of $A$ gives us $\relatebr{A}{(\sigma_0 \dt \sigma_1)}{\dvg}$, and so we get the desired
$\relatebr{\bigcup \A}{(\sigma_0 \dt \sigma_1)}{\dvg}$. For the backward direction, suppose that 
$\relatebr{\bigcup \A}{(\sigma_0 \dt \sigma_1)}{\dvg}$. Then $\relatebr{A}{(\sigma_0 \dt \sigma_1)}{\dvg}$ for some $A \in \A$. Thus 
Termination Equivalence of $A$ gives us $\relatebr{A}{\sigma_0}{\dvg}$, and so we get the desired
$\relatebr{\bigcup \A}{\sigma_0}{\dvg}$.

For the Forwards Frame Property, suppose that $\sigma_0 \# \sigma_1$ and $\relatebr{\bigcup \A}{\sigma_0}{\sigma_0'}$.
Then $\relatebr{A}{\sigma_0}{\sigma_0'}$ for some $A \in \A$, and so by the Forwards Frame Property of $A$, we have
$\sigma_0' \# \sigma_1$ and $\relatebr{A}{(\sigma_0 \dt \sigma_1)}{(\sigma_0' \dt \sigma_1)}$, which in turn
implies the desired result.

For the Backwards Frame Property, suppose that $\lnot \relatebr{\bigcup \A}{\sigma_0}{\fault}$ and
$\relatebr{\bigcup \A}{(\sigma_0 \dt \sigma_1)}{\sigma'}$. Then $\relatebr{A}{(\sigma_0 \dt \sigma_1)}{\sigma'}$
for some $A \in \A$, and for all $A \in \A$ we have $\lnot \relatebr{A}{\sigma_0}{\fault}$. Hence the
Backwards Frame Property of $A$ tells us that $\sigma' = \sigma_0' \dt \sigma_1$ and 
$\relatebr{A}{\sigma_0}{\sigma_0'}$, which implies the desired result.
\end{proof}
\fi

%%%%%%%%%%%%%%%%%%%%%%%%%%%%%%%%%%%%%%%%%%%%%%%%%%%%%%%%%%%%%%%%%%%%%%%%%%%%%%%%%%%%%%%%%%%%%%%%%%%%
\begin{figure}[t]
\begin{mathpar}
\begin{bnf}[r@{\ \ \ }c@{\ }]
\production{& C}
    \prodcase{c}
    \prodcase{C_1;C_2}
    \prodcase{C_1+C_2}
    \prodcase{C^*}
\end{bnf}
\end{mathpar}
\vspace*{-4ex}
\begin{align*}
\forall c \such \den{c} & \in \locact & \den{C_1;C_2} & \isdef \den{C_1};\den{C_2} \\
\den{C_1+C_2} & \isdef \den{C_1} \cup \den{C_2} & \den{C^*} & \isdef \bigcup_{n \in \N} \den{C}^n \\
\den{C}^0 & \isdef \id & \den{C}^{n+1} & \isdef \den{C};\den{C}^n
\end{align*}
\vspace*{-4ex}
\caption{Command Definition and Denotational Semantics}
\label{commands}
\end{figure}
%%%%%%%%%%%%%%%%%%%%%%%%%%%%%%%%%%%%%%%%%%%%%%%%%%%%%%%%%%%%%%%%%%%%%%%%%%%%%%%%%%%%%%%%%%%%%%%%%%%%

Figure~\ref{commands} defines our abstract program syntax and semantics. The language consists of primitive commands, sequencing ($C_1;C_2$),
nondeterministic choice ($C_1 + C_2$), and finite iteration ($C^*$). The semantics of primitive commands are abstracted~--- the only 
requirement is that they are local actions. Therefore, from the two previous lemmas and the trivial fact that $\id$ is a local action, 
it is clear that the semantics of \emph{every} program is a local action.

Note that our concrete language in Section~\ref{concrete} used if statements 
and while loops. As shown in~\cite{coy07}, it is possible to
represent if and while constructs with finite iteration and nondeterministic choice by including a primitive command $\assume(B)$,
which does nothing if the boolean expression $B$ is true, and diverges otherwise. 
\ifextended
Given this setup, we can define the primitive command $\assume(B)$ as follows:
\begin{align*}
\den{\assume & (B)} \isdef \{(\fault,\fault),(\dvg,\dvg)\} \,\, \cup \\
& \{(\sigma,\sigma) \pipe \den{B}\sigma = \true\} \cup \{(\sigma,\dvg) \pipe \den{B}\sigma = \false\} \,\, \cup \\
& \{(\sigma,\fault) \pipe \den{B}\sigma \text{ undefined}\}
\end{align*}
It is a simple matter to show that this is a local action. We can then syntactically define if and while statements as follows:
\begin{align*}
\condfull{B}{C_1}{C_2} & \isdef (\assume(B);C_1) + (\assume(\lnot B);C_2) \\
\while{B}{C} & \isdef (\assume(B);C)^*;\assume(\lnot B)
\end{align*}
Technically, these definitions only correctly implement if and while statements in terms of which states they can
terminate at~--- they do not correctly implement divergence behavior since they allow for arbitrary divergence. 
Thus these definitions should only be used if we do not care about divergence behavior. It is certainly
still possible to define fully correct if and while statements, but the technical details are
outside the scope of this work.
\fi

%%%%%%%%%%%%%%%%%%%%%%%%%%%%%%%%%%%%%%%%%%%%%%%%%%%%%%%%%%%%%%%%%%%%%%%%%%%%%%%%%%%%%%%%%%%%%%%%%%%%
\begin{figure}[t]
\begin{mathpar}
\inferrule*[right=\scriptsize(PRIM)]
{\lnot \relate{\den{c}}{\sigma}{\fault}}
{\assertd{\{\sigma\}}{c}{\{\sigma' \pipe \relate{\den{c}}{\sigma}{\sigma'}\}}}

\inferrule*[right=\scriptsize(SEQ)]
{\assertd{P}{C_1}{Q} \\
\assertd{Q}{C_2}{R}}
{\assertd{P}{C_1;C_2}{R}}

\inferrule*[right=\scriptsize(PLUS)]
{\assertd{P}{C_1}{Q} \\
\assertd{P}{C_2}{Q}}
{\assertd{P}{C_1+C_2}{Q}}

\inferrule*[right=\scriptsize(STAR)]
{\assertd{P}{C}{P}}
{\assertd{P}{C^*}{P}}

\inferrule*[right=\scriptsize(FRAME)]
{\assertd{P}{C}{Q}}
{\assertd{P*R}{C}{Q*R}}

\inferrule*[right=\scriptsize(CONSEQ)]
{P' \subseteq P \\
\assertd{P}{C}{Q} \\
Q \subseteq Q'}
{\assertd{P'}{C}{Q'}}

\inferrule*[right=\scriptsize(DISJ)]
{\forall i \in I \such \assertd{P_i}{C}{Q_i}}
{\assertd{\bigcup P_i}{C}{\bigcup Q_i}}

\inferrule*[right=\scriptsize(CONJ)]
{\forall i \in I \such \assertd{P_i}{C}{Q_i} \\
I \neq \emp}
{\assertd{\bigcap P_i}{C}{\bigcap Q_i}}
\end{mathpar}

\caption{Inference Rules}
\label{infrulesabs}
\end{figure}
%%%%%%%%%%%%%%%%%%%%%%%%%%%%%%%%%%%%%%%%%%%%%%%%%%%%%%%%%%%%%%%%%%%%%%%%%%%%%%%%%%%%%%%%%%%%%%%%%%%%

Now that we have defined the interpretation of programs as local actions, we can talk about the meaning of a
triple $\assert{P}{C}{Q}$. We define an assertion $P$ to be a set of states, and we say that a state
$\sigma$ satisfies $P$ iff $\sigma \in P$. We can then define the separating conjunction as follows:
\begin{align*}
P*Q \isdef \{\sigma \in \Sigma \pipe \exists \sigma_0 \in P,\sigma_1 \in Q \such \sigma = \sigma_0 \dt \sigma_1\}
\end{align*}

Given an assignment of primitive commands to local actions, we say that a triple is valid, written $\assertm{P}{C}{Q}$, just
when the following two properties hold for all states $\sigma$ and $\sigma'$:
\begin{align*}
& \text{1.) } \sigma \in P \Longrightarrow \lnot \relate{\den{C}}{\sigma}{\fault} \\
& \text{2.) } \sigma \in P \land \relate{\den{C}}{\sigma}{\sigma'} \Longrightarrow \sigma' \in Q
\end{align*}

The inference rules of the logic are given in Figure~\ref{infrulesabs}. Note that we are taking a significant presentation shortcut 
here in the inference rule for primitive commands. Specifically, we assume that we know the exact local action $\den{c}$ of each 
primitive command $c$. This assumption makes sense when we define our own primitive commands, as we do in the
logic of Section~\ref{concrete}. However, in a more general setting, we might be provided with an opaque function
along with a specification (precondition and postcondition) for the function. Since the function is opaque, we must
consider it to be a primitive command in the abstract setting. Yet we do not know how it is implemented, so we do not know 
its precise local action. In~\cite{coy07}, the authors provide a method for inferring a ``best'' local action
from the function's specification. With a decent amount of technical development, we can do something similar here,
using our stronger definition of locality. 
\begin{comment}
\ifextended
We will now present this technical development.

\begin{definition}
An \emph{axiom} is a triple of the form $\assert{p}{c}{q}$, where $c$ is a primitive command. 
Given a particular denotational semantics, we say that an axiom $\assert{p}{c}{q}$ is \emph{semantically valid} 
under that denotational semantics iff $\assertm{p}{c}{q}$. 
Given a set of axioms $Ax$, we define the judgment $\assertma{Ax}{p}{C}{q}$ to mean that whenever all the axioms 
in $Ax$ are semantically valid, $\assertm{p}{C}{q}$ holds. Also, given an axiom set $Ax$, we define the notation 
$Ax_c$ to be the set of all axioms in $Ax$ of the form $\assert{p}{c}{q}$ for some $p$ and $q$.
\end{definition}

Once we have a set of axioms $Ax_c$, we need to come up with an interpretation of those axioms as a
behavior-preserving action that makes all the axioms semantically valid (such an action is referred to as the
\emph{best local action} in~\cite{coy07}). Unfortunately, we find that with our strong form of locality, 
there does not always exist such an action (excepting, of course, the trivial action that never terminates on
any input).

For example, consider a situation where we have the two axioms $A_1: \assert{x=1}{c}{x=2}$ and
$A_2: \assert{y=1}{c}{y=2}$. Let $\sigma_1$ be the variable store mapping $x$ to $1$, and $\sigma_2$ be the store mapping
$y$ to $1$. Then $A_1$ tells us that it is safe to run $c$ on $\sigma_1$, and the result should be that the value of $x$
is updated to $2$. Similarly, $A_2$ implies that $c$ can also run safely on $\sigma_2$, updating $y$. So
if we let $f$ denote the local, domain-preserving action that we wish to come up with, we have that
$f(\sigma_1) \subseteq \{\sigma_1'\}$ and $f(\sigma_2) \subseteq \{\sigma_2'\}$ where $\sigma_1'$ is the
variable store mapping $x$ to $2$, and $\sigma_2'$ is the one mapping $y$ to $2$. Furthermore, since we wish
to come up with a non-trivial action, we can say that $\sigma_1'$ and $\sigma_2'$ are actually possible
termination states~--- that is, $f(\sigma_1) = \{\sigma_1'\}$ and $f(\sigma_2) = \{\sigma_2'\}$. But
now locality tells us that $f(\sigma_1 \bullet \sigma_2) = \{\sigma_1' \bullet \sigma_2\}$, and that
$f(\sigma_1 \bullet \sigma_2) = \{\sigma_2' \bullet \sigma_1\}$. Hence 
$\sigma_1' \bullet \sigma_2 = \sigma_2' \bullet \sigma_1$, which is clearly false. Thus we conclude
that no desired non-trivial action exists.

Notice that this example does not give a contradiction when using the weaker form of locality, since
we are then free to set $f(\sigma_1 \bullet \sigma_2) = \emp$ (meaning that the action never terminates
when run on the combined state). This is precisely what is done in~\cite{coy07} to deal with the situation.

Fundamentally, the reason why the above example causes a problem is that the axioms specify incompatible 
behaviors on two substates on some particular state. We will therefore resolve the problem by
requiring that our axioms never do this. Specifically, consider a situation where we have axioms $\assert{p_1}{c}{q_1}$
and $\assert{p_2}{c}{q_2}$, and states $\sigma, \sigma_1, \sigma_2$ such that $\sigma_1 \preceq \sigma$,
$\sigma_2 \preceq \sigma$, $\sigma_1 \in p_1$, and $\sigma_2 \in p_2$. Let $\bar{\sigma}_1$ and $\bar{\sigma}_2$
be the states such that $\sigma_1 \bullet \bar{\sigma}_1 = \sigma_2 \bullet \bar{\sigma}_2 = \sigma$.
Then we interpret the first axiom as saying that $f(\sigma) = q_1^{dom(\sigma_1)} * \{\bar{\sigma}_1\}$, and
similarly the second axiom says that $f(\sigma) = q_2^{dom(\sigma_2)} * \{\bar{\sigma}_2\}$. We therefore
need to require that $q_1^{dom(\sigma_1)} * \{\bar{\sigma}_1\} = q_2^{dom(\sigma_2)} * \{\bar{\sigma}_2\}$.
The formal definition of this property follows.

\vspace*{0.5ex}
\begin{definition}
Given that $\sigma_0 \preceq \sigma$, define the notation $\sigma - \sigma_0$ to be the unique state $\sigma_1$ such that
$\sigma = \sigma_0 \bullet \sigma_1$ (uniqueness is implied by the cancellativity property of separation algebras).
Then, given set of axioms $Ax$, we say that $Ax$ is \emph{permissible} if, for any primitive command $c$, and any (not necessarily distinct) axioms 
$\assert{p_1}{c}{q_1}$ and $\assert{p_2}{c}{q_2}$ in $Ax_c$, and any states $\sigma, \sigma_1, \sigma_2$, the following holds:
\[\begin{array}{l} \sigma_1 \preceq \sigma \land \sigma_2 \preceq \sigma \land \sigma_1 \in p_1 \land \sigma_2 \in p_2 \\
\hspace{5mm} \Longrightarrow q_1^{dom(\sigma_1)} * \{\sigma - \sigma_1\} = q_2^{dom(\sigma_2)} * \{\sigma - \sigma_2\} \end{array}\]
We require that all instantiations of the abstract logic use permissible axiom sets.
\end{definition}
\else
\end{comment}
These details can be found in the technical report of our APLAS publication~\cite{costanzo12:bpsltr}.
%\fi

Given this setup, we can now prove soundness and completeness of our 
abstract logic. The details of the proof can be found in our Coq implementation~\cite{costanzo-thesis}.
\begin{thm}[Soundness and Completeness]
\begin{align*}
\assertd{P}{C}{Q} \iff \assertm{P}{C}{Q}
\end{align*}
\end{thm}
