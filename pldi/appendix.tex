\section{Appendix: Proof of Theorem~\ref{end-to-end}}
\label{appendix}

We present the full proof for transferring security across simulation here,
even when the type of observations changes across machines. The definitions
of machines and high-level security are identical to those presented in 
Section~\ref{methodology}. The definition of low-level security is also
the same, but we need to define whole-execution behaviors more formally.

\begin{definition}[Behavioral State]
Given a machine $M$ and a partial order $\preceq$ over the observation 
type $\Omega_M$, we say that a program state $\sigma$ is behavioral
for principal $l$, written $\behaviorals{M}{l}{\sigma}$,
if all executions starting from $\sigma$ respect 
the partial order; i.e., the following monotonicity property holds:
\[\quad \forall \sigma' \such \sigma \steprel^* \sigma' 
\Longrightarrow \observe{l}{\sigma} \preceq \observe{l}{\sigma'}\]
\end{definition}

\begin{definition}[Behavioral Machine]
We say that a machine $M$ is behavioral for principal $l$, written 
$\behavioral{M}{l}$, when the machine has the following components:
\begin{itemize}
\item a partial order $\preceq$ over the observation type $\Omega_M$
\item a proof that all states of $M$ are behavioral for $l$
\end{itemize}
\end{definition}

We will give a semiformal definition of whole-execution behaviors here. 
The formal Coq definition involves a
combination of inductive and coinductive types (to handle behaviors of
both terminating and non-terminating executions). Note that our 
definition is quite similar to the one used in CompCert~\cite{Leroy-backend},
except that we use state observations as the basic building block, while 
CompCert uses \emph{traces}, which are input/output events labeled on 
transitions.
\begin{definition}[Whole-Execution Behaviors]
Given a machine $M$ with a partial order defined over $\Omega_M$,
and a state $\sigma$ that is behavioral for 
principal $l$, we write $\behavior{M;l}{\sigma}$ to represent the (potentially 
infinite) set of whole-execution behaviors that can arise from some execution 
of $M$ starting from $\sigma$. The behaviors (elements of this set) can 
be one of four kinds: \emph{fault}, \emph{termination}, \emph{silent divergence},
and \emph{reactive divergence}. In the following, variable $o$ ranges over 
observations and $os$ ranges over infinite streams of observations:
\begin{enumerate}
\item $\ttt{Fault}(o) \in \behavior{M;l}{\sigma}$ indicates that there
is an execution $\sigma~\steprel^*~\sigma'$ where $\sigma'$ is not
a final state, $\sigma'$ cannot take a step to any state, and 
$o = \observe{l}{\sigma'}$.
\item $\ttt{Term}(o) \in \behavior{M;l}{\sigma}$ indicates that there
is an execution $\sigma~\steprel^*~\sigma'$ where $\sigma'$ is 
a final state and $o = \observe{l}{\sigma'}$.
\item $\ttt{Silent}(o) \in \behavior{M;l}{\sigma}$ indicates that there
is an execution $\sigma~\steprel^*~\sigma'$ where $o = \observe{l}{\sigma'}$
and there is an infinite execution starting from $\sigma'$ for which all
states in that infinite execution have identical observations.
\item $\ttt{React}(os) \in \behavior{M;l}{\sigma}$ indicates that there
is an infinite execution starting from $\sigma$ that ``produces'' each of the
infinitely-many observations of $os$ in order. An observation $o$ is ``produced'' 
in an execution when there exists some single step in the execution 
$\sigma'~\steprel~\sigma''$ with $o = \observe{l}{\sigma''}$ and 
$\observe{l}{\sigma'} \neq \observe{l}{\sigma''}$.
\end{enumerate}
\end{definition}

Next we define simulations that can relate machines with different 
observation types. As a first attempt, we start with the same 
simulation definition as presented in Section~\ref{methodology}, but 
we introduce a second relation that relates observations of one machine 
to observations of the other.

\begin{definition}[Generalized Simulation, Attempt 1]
Given two machines $M$, $m$, a principal $l$, a relation $R$ between
states of $M$ and states of $m$, and a relation $\obsrelr{R}$ between
observations of $M$ and observations of $m$, we say that there is a
simulation from $M$ to $m$ using $R$ and $\obsrelr{R}$,
written $\simu{M}{m}{R;\obsrelr{R};l}$, when:
{\small
\begin{align*}
& 1.) \quad \forall \sigma, \sigma' \in \Sigma_M, \,\, s \in \Sigma_m \such \\
& \qquad \quad \sigma \steprel \sigma' \land R(\sigma,s) \\
& \qquad \quad \Longrightarrow 
\exists s' \in \Sigma_m \such s \steprel^* s' \land R(\sigma',s') \\
& 2.) \quad \forall \sigma \in \Sigma_M, \,\, s \in \Sigma_m \such \\
& \qquad \quad \sigma \in F_M \land R(\sigma,s) \Longrightarrow s \in F_m \\
& 3.) \quad \forall \sigma \in \Sigma_M, \,\, s \in \Sigma_m \such \\
& \qquad \qquad R(\sigma,s) \Longrightarrow \obsrelr{R}(\observem{M}{l}{\sigma},\observem{m}{l}{s}) \\
& 4.) \quad \forall \sigma_1, \sigma_2 \in \Sigma_M, \,\, s_1, s_2 \in \Sigma_m \such \\
& \qquad \qquad \observem{M}{l}{\sigma_1} = \observem{M}{l}{\sigma_2} \\
& \qquad \qquad \land \obsrelr{R}(\observem{M}{l}{\sigma_1},\observem{m}{l}{s_1}) \\
& \qquad \qquad \land \obsrelr{R}(\observem{M}{l}{\sigma_2},\observem{m}{l}{s_2}) \\
& \qquad \qquad \Longrightarrow \observem{m}{l}{s_1} = \observem{m}{l}{s_2}
\end{align*}}
\end{definition}

\noindent{}The first two properties are identical to those of the simulation
definition presented in Section~\ref{methodology}, while the third and
fourth properties are a simplification of the indistinguishability
preservation property. Notice that these two properties together trivially 
imply indistinguishability preservation; thus this definition is a
stricter form of simulation.

It turns out that we can simplify this definition significantly. The fourth
property above actually just states that $\obsrelr{R}$ always relates equal
inputs to equal outputs; i.e., $\obsrelr{R}$ can only relate a high-level
observation to at most one low-level observation. If we enforce a minor 
requirement that $\obsrelr{R}$ always relates a high-level observation to
\emph{at least} one low-level observation, then we can reformulate the relation
as a function, resulting in a significantly simpler simulation definition:

\begin{definition}[Generalized Simulation]
\label{gensimdef}
Given two machines $M$, $m$, a principal $l$, a relation $R$ between
states of $M$ and states of $m$, and a function $\obsrel{R}$ from
observations of $M$ to observations of $m$, we say that there is a
simulation from $M$ to $m$ using $R$ and $\obsrel{R}$,
written $\simu{M}{m}{R;\obsrel{R};l}$, when:
{\small
\begin{align*}
& 1.) \quad \forall \sigma, \sigma' \in \Sigma_M, \,\, s \in \Sigma_m \such \\
& \qquad \quad \sigma \steprel \sigma' \land R(\sigma,s) \\
& \qquad \quad \Longrightarrow 
\exists s' \in \Sigma_m \such s \steprel^* s' \land R(\sigma',s') \\
& 2.) \quad \forall \sigma \in \Sigma_M, \,\, s \in \Sigma_m \such \\
& \qquad \quad \sigma \in F_M \land R(\sigma,s) \Longrightarrow s \in F_m \\
& 3.) \quad \forall \sigma \in \Sigma_M, \,\, s \in \Sigma_m \such \\
& \qquad \qquad R(\sigma,s) \Longrightarrow 
\obsrel{R}(\observem{M}{l}{\sigma}) = \observem{m}{l}{s}
\end{align*}}
\end{definition}

We next define a bisimulation, and state a lemma that reformulates 
high-level security as a bisimulation (the proof is omitted here, but
it is completely straightforward).

\begin{definition}[Bisimulation]
\label{bisimdef}
Given two machines $M$, $m$, a principal $l$, a relation $R$ between
states of $M$ and states of $m$, and an \emph{invertible} function 
$\obsrel{R}$ from observations of $M$ to observations of $m$, 
we say that there is a bisimulation from $M$ to $m$ using $R$ and 
$\obsrel{R}$, written $\bisimu{M}{m}{R;\obsrel{R};l}$, when
$\simu{M}{m}{R;\obsrel{R};l}$ and $\simu{m}{M}{R^{-1};\obsrel{R}^{-1};l}$.
\end{definition}

\begin{definition}[Invariant-Aware Indistinguishability]
{\small
\begin{align*}
& \Theta^{I}_l(\sigma_1,\sigma_2) \isdef 
\sigma_1 \in I \land \sigma_2 \in I \land
\observe{l}{\sigma_1} = \observe{l}{\sigma_2}
\end{align*}}
\end{definition}

\begin{lem}[High-Level Security Bisimulation]
\label{secure-bisim}
{\small
\begin{align*}
& \forall M, I, l \such \secure{M}{I}{l} \Longrightarrow 
\bisimu{M}{M}{\Theta^{I}_l;\mathid;l}
\end{align*}}
\end{lem}

The next step is to define the method of comparing sets of behaviors
between machines that have different observation types.
In the following, we will overload the $\obsrel{R}$ function to apply
to behaviors in the obvious way 
(e.g., $\obsrel{R}(\ttt{Term}(o)) = \ttt{Term}(\obsrel{R}(o))$).
Given a simulation $\simu{M}{m}{R;\obsrel{R};l}$, with both $M$ and
$m$ behavioral for $l$, we define subset and equality relations 
between sets of behaviors by applying $\obsrel{R}$ to every element 
of the first set:
\begin{definition}[Behavior Subset]
\label{beh-subset}
{\small
\begin{align*}
& \simu{\behavior{M;l}{\sigma}}{\behavior{m;l}{s}}{\obsrel{R}}
\isdef \\
& \qquad \forall b \such b \in \behavior{M;l}{\sigma} \Longrightarrow
\obsrel{R}(b) \in \behavior{m;l}{s}
\end{align*}}
\end{definition}

\begin{definition}[Behavior Equality]
\label{beh-equality}
{\small
\begin{align*}
& \bisimu{\behavior{M;l}{\sigma}}{\behavior{m;l}{s}}{\obsrel{R}}
\isdef \\
& \qquad \forall b \such b \in \behavior{M;l}{\sigma} \iff
\obsrel{R}(b) \in \behavior{m;l}{s}
\end{align*}}
\end{definition}

\noindent
Note that this equality relation is asymmetric since $\obsrel{R}$ can only
apply to behaviors in the first set, and $\obsrel{R}$ is not guaranteed
to have an inverse.

We now state some important lemmas relating behaviors and simulations,
omitting proofs here. For the most part, the proofs are reasonably 
straightforward, requiring some standard case analyses, as well as 
applications of induction and coinduction hypotheses. We begin with
notations for the standard properties of safety and determinism, and
then we state the various lemmas using these notations.

\begin{definition}[Safety]
We say that a machine $M$ is safe under state predicate $I$, 
written $\safe{M}{I}$, when the following progress and
preservation properties hold:
{\small
\begin{align*}
& 1.) \quad \forall \sigma \in I - F_M \such
\exists \sigma' \such \sigma \steprel \sigma' \\
& 2.) \quad \forall \sigma, \sigma' \such
\sigma \in I \land
\sigma \steprel \sigma' \Longrightarrow \sigma' \in I
\end{align*}}
\end{definition}

\begin{definition}[Determinism]
We say that a machine $M$ is deterministic, written $\determ{M}$,
when the following properties hold:
{\small
\begin{align*}
& 1.) \quad \forall \sigma, \sigma', \sigma'' \such
\sigma \steprel \sigma' \land \sigma \steprel \sigma''
\Longrightarrow \sigma' = \sigma'' \\
& 2.) \quad \forall \sigma \in F_M \such
\lnot \, \exists \sigma' \such \sigma \steprel \sigma'
\end{align*}}
\end{definition}

\begin{lem}[Behavior Exists]
\label{beh-exists}
{\small
\begin{align*}
& \forall M, l, \sigma \such \behaviorals{M}{l}{\sigma}
\Longrightarrow
\behavior{M;l}{\sigma} \neq \emptyset
\end{align*}}
\end{lem}

\begin{lem}[Behavior Determinism]
\label{beh-det}
{\small
\begin{align*}
& \forall M, l, \sigma \such 
\behaviorals{M}{l}{\sigma} \land \determ{M} \Longrightarrow
| \behavior{M;l}{\sigma} | = 1
\end{align*}}
\end{lem}

\begin{comment}
\begin{lem}[Behavior Safety]
\begin{align*}
& \forall M, I, l, \sigma \such \\
& \qquad \behaviorals{M}{l}{\sigma} \land \safe{M}{I} \land \sigma \in I \\
& \qquad \Longrightarrow 
\forall o \such \ttt{Fault}(o) \notin \behavior{M;l}{\sigma}
\end{align*}
\end{lem}
\end{comment}

\begin{lem}[Simulation Implies Behavior Subset]
\label{beh-sim-subset}
{\small
\begin{align*}
& \forall M, m, I, R, 
\obsrel{R},
l, \sigma, s \such \\
& \qquad
\behaviorals{M}{l}{\sigma} \land \behaviorals{m}{l}{s} \\
& \qquad \land \safe{M}{I} \land 
\simu{M}{m}{R;\obsrel{R};l}
\land \sigma \in I \land R(\sigma,s) \\
& \qquad \Longrightarrow 
\simu{\behavior{M;l}{\sigma}}{\behavior{m;l}{s}}{\obsrel{R}}
\end{align*}}
\end{lem}

\begin{lem}[Bisimulation Implies Behavior Equality]
\label{beh-bisim}
{\small
\begin{align*}
& \forall M, m, R, 
\obsrel{R},
l, \sigma, s \such \\
& \qquad
\behaviorals{M}{l}{\sigma} \land \behaviorals{m}{l}{s} \\
& \qquad \land 
\bisimu{M}{m}{R;\obsrel{R};l} \land R(\sigma,s) \\
& \qquad \Longrightarrow 
\bisimu{\behavior{M;l}{\sigma}}{\behavior{m;l}{s}}{\obsrel{R}}
\end{align*}}
\end{lem}

\noindent
Finally, we state\cut{ and prove} the end-to-end security theorem. \cut{
We
assume there is a specification machine $M$ that is secure for 
principal $l$ under invariant $I$, a deterministic implementation 
machine $m$ that is behavioral for $l$, and a simulation from 
$M$ to $m$. The theorem then says that
if we take any two indistinguishable states $\sigma_1$,
$\sigma_2$ of $M$ (which satisfy $I$), and relate them down
to two states $s_1$, $s_2$ of $m$, then the whole-execution
behaviors of $s_1$ and $s_2$ are equal. Note that we require
the implementation to be deterministic~--- this is a reasonable
requirement since the implementation is supposed to represent
the real machine execution, and real machines are deterministic.
Crucially, the specification may still be nondeterministic.

}Since the statement of this theorem is a bit unwieldy, we 
will reformulate it below to directly express that
high-level security implies low-level security (like we
do in Section~\ref{methodology}).

\begin{thm}[Generalized End-to-End Security]
{\small
\begin{align*}
& \forall M, m, I, R, 
\obsrel{R},
l, \sigma_1, \sigma_2, s_1, s_2 \such \\
& \qquad \secure{M}{I}{l} \land \behavioral{m}{l} \land \determ{m} \land
\simu{M}{m}{R;\obsrel{R};l} \\
& \qquad \land \sigma_1 \in I \land \sigma_2 \in I \land
\observem{M}{l}{\sigma_1} = \observem{M}{l}{\sigma_2} \\
& \qquad \land R(\sigma_1,s_1) \land R(\sigma_2,s_2) \\
& \qquad \Longrightarrow
\behavior{m;l}{s_1} = \behavior{m;l}{s_2}
\end{align*}}
\end{thm}

\begin{proof}
We prove this theorem by defining a new machine $N$ in between $M$
and $m$, and proving simulations from $M$ to $N$ and from $N$
to $m$. $N$ will mimic $M$ in terms of program states
and transitions, while it will mimic $m$ in terms of observations.
More formally, we define $N$ to have the following components:
\begin{itemize}
\item program state $\Sigma_M$
\item initial states $I_M$
\item final states $F_M$
\item transition relation $T_M$
\item observation type $\Omega_m$
\item observation function 
$\observem{N}{l}{\sigma} \isdef \obsrel{R}(\observem{M}{l}{\sigma})$
\end{itemize}

First, we establish the simulation $\simu{M}{N}{\mathid;\obsrel{R};l}$.
Referring to Definition~\ref{gensimdef}, the first two properties hold
trivially since $N$ has the same transition relation and final
state set as $M$. The third property reduces to exactly our
definition of $\observem{N}{l}{-}$ given above.

Next, we establish the simulation $\simu{N}{m}{R;\mathid;l}$. The
first two properties of Definition~\ref{gensimdef} are exactly
the same as the first two properties of the provided simulation
$\simu{M}{m}{R;\obsrel{R};l}$, and thus they hold. For the third
property, assuming we know $R(\sigma,s)$, we have 
$\mathid(\observem{N}{l}{\sigma}) = 
\observem{N}{l}{\sigma} = \obsrel{R}(\observem{M}{l}{\sigma})
= \observem{m}{l}{s}$, where the final equality comes from
the third property of the provided simulation.

The next step is to relate the behaviors of $N$ with those of
$m$. In order to do this, we first must show that $N$ has 
well-defined behaviors for executions starting from $\sigma_1$
or $\sigma_2$. In other words, we must prove 
$\behaviorals{N}{l}{\sigma_1}$ and $\behaviorals{N}{l}{\sigma_2}$.
We will focus on the proof for $\sigma_1$; the other proof is
analogous. We use the same partial order as provided
by the assumption $\behavioral{m}{l}$.
Consider any execution $\sigma_1~\steprel^*~\sigma_1'$ in $N$.
Since $R(\sigma_1,s_1)$, we can use the simulation 
$\simu{N}{m}{R;\mathid;l}$ established above, yielding an execution
$s_1~\steprel^*~s_1'$, for some $s_1'$ (technically, the simulation 
property only applies to single steps in the higher machine; 
however, it can easily be extended to multiple steps
through induction on the step relation). Additionally, we have 
$R(\sigma_1,s_1)$ and $R(\sigma_1',s_1')$, implying by the third property 
of simulation that $\observem{N}{l}{\sigma_1} = \observem{m}{l}{s_1}$ and 
$\observem{N}{l}{\sigma_1'} = \observem{m}{l}{s_1'}$.
Since $m$ is behavioral, we also have 
$\observem{m}{l}{s_1}~\preceq~\observem{m}{l}{s_1'}$. Hence we conclude
$\observem{N}{l}{\sigma_1}~\preceq~\observem{N}{l}{\sigma_1'}$, as desired.

We now know that $\behaviorals{N}{l}{\sigma_1}$ and 
$\behaviorals{N}{l}{\sigma_2}$. Notice that
when $\obsrel{R}$ is $\mathid$, our definitions of behavior subset and
equality (Definitions~\ref{beh-subset} and~\ref{beh-equality}) reduce
to standard subset and set equality. Therefore, applying 
Lemma~\ref{beh-sim-subset} to the established simulation
$\simu{N}{m}{R;\mathid;l}$ tells us that 
$\behavior{N;l}{\sigma_1}~\subseteq~\behavior{m;l}{s_1}$
and $\behavior{N;l}{\sigma_2}~\subseteq~\behavior{m;l}{s_2}$ (note that
the safety precondition of Lemma~\ref{beh-sim-subset} holds because $M$ and $N$
have the same state type and transition relation). Furthermore,
since $m$ is deterministic, Lemma~\ref{beh-det} gives us 
$| \behavior{m;l}{s_1} | = | \behavior{m;l}{s_2} | = 1$. Since 
Lemma~\ref{beh-exists} guarantees that neither $\behavior{N;l}{\sigma_1}$ nor
$\behavior{N;l}{\sigma_2}$ is empty, we conclude that
$\behavior{N;l}{\sigma_1} = \behavior{m;l}{s_1}$
and $\behavior{N;l}{\sigma_2} = \behavior{m;l}{s_2}$.

To complete the proof, we now just need to show that 
$\behavior{N;l}{\sigma_1} = \behavior{N;l}{\sigma_2}$. Recall
Lemma~\ref{secure-bisim}, which expresses a security proof as 
a bisimulation. Extending that lemma with an observation relation
of $\mathid$, we have the bisimulation $\bisimu{M}{M}{\Theta^{I}_l;\mathid;l}$.
We would like to apply Lemma~\ref{beh-bisim}, but we first need
to convert this bisimulation into one on $N$, since $M$ is not
behavioral. Since $M$ and $N$ share program state type, final states,
and transition relation, it is not difficult to see that the first
two required properties of the simulation $\simu{N}{N}{\Theta^{I}_l;\mathid;l}$ 
hold. If we can establish the third property, then we will obtain the
desired bisimulation $\bisimu{N}{N}{\Theta^{I}_l;\mathid;l}$ since $\mathid$ 
is obviously invertible and $\Theta^{I}_l$ is symmetric. The third
property requires us to prove that 
$\Theta^{I}_l(\sigma_1,\sigma_2) \Longrightarrow 
\observem{N}{l}{\sigma_1} = \observem{N}{l}{\sigma_2}$.
By definition, $\Theta^{I}_l(\sigma_1,\sigma_2)$ implies that
$\observem{M}{l}{\sigma_1} = \observem{M}{l}{\sigma_2}$.
Notice that $\observem{N}{l}{\sigma_1} = \observem{N}{l}{\sigma_2}$
following from this fact is exactly the indistinguishability 
preservation property we discussed earlier. Indeed, utilizing the 
third property of the established simulation 
$\simu{M}{N}{\mathid;\obsrel{R};l}$, we
have $\observem{N}{l}{\sigma_1} = \obsrel{R}(\observem{M}{l}{\sigma_1}) = 
\obsrel{R}(\observem{M}{l}{\sigma_2}) = \observem{N}{l}{\sigma_2}$.

Finally, we instantiate Lemma~\ref{beh-bisim} with both machines
being $N$. Notice that we technically need to establish the
precondition of $\Theta^{I}_l(\sigma_1,\sigma_2)$ before we can 
apply the lemma. By definition, this reduces to 
$\sigma_1~\in~I \land \sigma_2~\in~I \land 
\observem{N}{l}{\sigma_1}~=~\observem{N}{l}{\sigma_2}$.
The first two facts hold by assumption, and the third one
follows from $\observem{M}{l}{\sigma_1} = \observem{M}{l}{\sigma_2}$ 
by exactly the same reasoning we just used above.
Now, applying Lemma~\ref{beh-bisim} yields the conclusion
$\bisimu{\behavior{N;l}{\sigma_1}}{\behavior{N;l}{\sigma_2}}{\mathid}$.
As mentioned earlier, behavior equality reduces to standard set
equality when $\obsrel{R}$ is $\mathid$, and so we get the desired
$\behavior{N;l}{\sigma_1} = \behavior{N;l}{\sigma_2}$, completing
the proof.
\end{proof}

To reformulate the theorem, we wrap up many of the preconditions
into a low-level indistinguishability relation. The relation
is parameterized by a high-level machine, a principal, a high-level
invariant, and a simulation relation. It says that two low-level
states are indistinguishable if we can find two indistinguishable
high-level states that are related via the simulation relation.

\begin{definition}[Low-Level Indistinguishability]
{\small
\begin{align*}
& \phi(M,l,I,R) \isdef \\
& \quad \lambda s_1, s_2 \such \exists \sigma_1, \sigma_2 \in I \such \\
& \quad \quad \observem{M}{l}{\sigma_1} = \observem{M}{l}{\sigma_2} 
\land R(\sigma_1,s_1) \land R(\sigma_2,s_2)
\end{align*}}
\end{definition}

\noindent
End-to-end security is now reformulated to explicitly say that
high-level security implies low-level security.

\begin{cor}[Generalized End-to-End Security, Final]
{\small
\begin{align*}
& \forall M, m, I, R, 
\obsrel{R},
l \such \\
& \qquad \behavioral{m}{l} \land \determ{m} \land 
\simu{M}{m}{R;\obsrel{R};l} \\
& \qquad \Longrightarrow 
\left(\secure{M}{I}{l} \Longrightarrow \bsecure{m}{\phi(M,l,I,R)}{l}\right)
\end{align*}}
\end{cor}





