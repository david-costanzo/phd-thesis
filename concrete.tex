\section{Impact on a Concrete Separation Logic}
\label{concrete}

In this section, we will demonstrate how behavior-preserving
locality can be enforced in a standard model of Separation Logic
without any negative impact on using the program logic. In the 
standard Reynolds' Separation Logic model~\cite{reynolds02}, a program state
consists of two components: a variable store and a heap. When new
memory is allocated, the memory is taken from outside the state
and added into the heap. As mentioned in Section~\ref{locality}, this
notion of memory allocation violates our Forwards Frame Property,
so we will instead include an
explicit free list inside the program state. Thus a state is now is a
triple $(s,h,f)$ consisting of a store, a heap, and a free list, with
the heap and free list occupying disjoint areas of
memory. Newly-allocated memory always comes from the free list,
while deallocated memory goes back into the free list. Since the
standard formulation of Separation Logic assumes that memory is
infinite and hence that allocation never fails, we similarly require
the free list to be infinite. More specifically, we require that
there is some location $n$ such that all locations above $n$ are in
the free list.
Formally, states are defined as follows:
{\small
\begin{align*}
& \text{Var } V \isdef \{x,y,z,\ldots \} \qquad \text{Store } S \isdef V \to \Z \qquad \text{Heap } H \isdef \N \finpf \Z \\
& \text{Free List } F \isdef \{N \in \pwrset{\N} \pipe \exists n \such \forall k \geq n \such k \in N\} \\
& \text{State } \Sigma \isdef \{(s,h,f) \in S \times H \times F \pipe \text{dom}(h) \cap f = \emp\}
\end{align*}
}
\indent{}As a point of clarification, we are not claiming here that including the free list in 
the state model is a novel idea. Other systems (e.g.,~\cite{rg09}) have made use of a 
very similar idea. The two novel contributions that we will show in this section are:
(1) that a state model which includes an explicit free list can provide a behavior-preserving
semantics, and (2) that the corresponding program logic can be made to be completely 
backwards-compatible with standard Separation Logic (meaning that any valid Separation 
Logic derivation is also a valid derivation in our logic).

%%%%%%%%%%%%%%%%%%%%%%%%%%%%%%%%%%%%%%%%%%%%%%%%%%%%%%%%%%%%%%%%%%%%%%%%%%%%
\begin{figure}[t]
\begin{mathpar}

\hspace{-15mm}    
\begin{bnf}[r@{\ \ \ }c@{\ }]

\production{& E}
    \prodcase{E + E'}
    \prodcase{E - E'}
    \prodcase{\ldots}
    \prodcase{-1}
    \prodcase{0}
    \prodcase{1}
    \prodcase{\ldots}
    \prodcase{x}
    \prodcase{y}
    \prodcase{\ldots}
    
\production{& B}
    \prodcase{E=E'}
    \prodcase{\false}
    \prodcase{B \Rightarrow B'}
    
\production{& P,Q}
    \prodcase{B}
    \prodcase{\false}
    \prodcase{\empa}
    \prodcase{E \mapsto E'}
    \prodcase{P \Rightarrow Q}
    \prodcase{P*Q}
    
\production{& C}
    \prodcase{\skp}
    \prodcase{x:=E}
    \prodcase{x:=[E]}
    \prodcase{[E]:=E'}
    \prodcase{x:=\cons(E_1,\ldots,E_n)}
    \prodcase{\free(E)}
    \prodcase{C;C'}
    \prodcase{\condfull{B}{C}{C'}}
    \prodcase{\while{B}{C}}
    
\end{bnf}
\end{mathpar}

\caption{Assertion and Program Syntax}
\label{syntax}
\end{figure}

%%%%%%%%%%%%%%%%%%%%%%%%%%%%%%%%%%%%%%%%%%%%%%%%%%%%%%%%%%%%%%%%%%%%%%%%%%%%

\ifextended
We adopt the following standard notations: $\dom{h}$ is the domain of the heap $h$;
$s[x \mapsto v]$ is the store which is identical to $s$, except that the value of variable $x$
is updated to $v$; $h[l \mapsto v]$ is the heap which is identical to $h$, except that location
$l$ is either added to $h$ with value $v$ if it does not exist in $h$, or updated with value $v$
if it does exist; $h \backslash l$ is the heap resulting from removing location $l$ from $h$;
$h_0 \# h_1$ is true just when $\dom{h_0}$ and $\dom{h_1}$ do not overlap;
$h_0 \dt h_1$ is equal to the union of $h_0$ and $h_1$ if $h_0 \# h_1$, and is undefined otherwise.
We also overload the disjointness (\#) operator to work with free lists~--- e.g., $h \# f$ says 
that $\dom{h}$ and $f$ are disjoint.
\fi

Assertion syntax and program syntax are given in Figure~\ref{syntax}, and are exactly
the same as in the standard model for Separation Logic. 
\ifextended
This syntax includes expressions
$E$ and boolean expressions $B$, both of which can be evaluated under a given variable store,
without any knowledge of the heap. These valuations are denoted by $\den{E}s$ and $\den{B}s$ 
for a given store $s$; the former
evaluates to an integer, while the latter evaluates to a boolean. Their formal definitions
are omitted here, but are straightforward and standard in the literature.
\fi

Our satisfaction judgement $(s,h,f) \models P$ for an assertion $P$ is
defined by ignoring the free list and only considering whether $(s,h)$
satisfies $P$. The definition of $(s,h)~\models~P$ is identical to
that of standard Separation Logic, and is given in Figure~\ref{satisfaction}.
The most important cases are $E \mapsto E'$ and $P*Q$. $E
\mapsto E'$ says that the current heap consists \emph{only} of the
memory cell at address $\den{E}s$, and that the cell at that address
maps to the value $\den{E'}s$. $P*Q$ says that we can separate the
current heap into two disjoint subheaps $h_0$ and $h_1$, with $h_0$
satisfying $P$ and $h_1$ satisfying $Q$. We also define the standard
syntactic sugars $E \mapsto E_0,\ldots,E_n$ to be $(E \mapsto E_0) *
\ldots * (E+n \mapsto E_n)$, and $E \mapsto -$ to be $\exists x . E
\mapsto x$ (where $x$ is not free in $E$).

\begin{figure}[t]

\begin{align*}
(s,h) \models B & \iff \den{B}s = \true \\
(s,h) \models \false & \iff \text{never} \\
(s,h) \models \empa & \iff \dom{h} = \emptyset \\
(s,h) \models E \mapsto E' & \iff \dom{h} = \{\den{E}s\} \land h(\den{E}s) = \den{E'}s \\
(s,h) \models P \Rightarrow Q & \iff \text{if } (s,h) \models P \text{, then } (s,h) \models Q \\
%%%%%%%%%%%%%%%%%%%%%%%%%%%%%%%%%
(s,h) \models P*Q & \iff 
\left(\begin{aligned}
\exists & h_0, h_1 \such h_0 \# h_1 \land h_0 \dt h_1 = h \hspace{1mm} \land \\
& (s,h_0) \models P \land (s,h_1) \models Q
\end{aligned}\right)
%%%%%%%%%%%%%%%%%%%%%%%%%%%%%%%%%
\end{align*}

\caption{Satisfaction of Assertions}
\label{satisfaction}
\end{figure}

%%%%%%%%%%%%%%%%%%%%%%%%%%%%%%%%%%%%%%%%%%%%%%%%%%%%%%%%%%%%%%%%%%%%%%%%%%%%

\ifextended

Figure~\ref{opsem} defines the small-step operational semantics for our 
machine. $x:=[E]$ and
$[E]:=E'$ correspond to reading from and writing to the heap, respectively. $x:=\cons(E_1,\ldots,E_n)$
allocates a nondeterministically-chosen contiguous block of $n$ heap cells from the free list.
The most interesting rules are those for allocation and deallocation, since they make use of the free list. 
Note that none of the operations make use of any memory existing outside the program state~---
this is the key for obtaining behavior-preservation.

\begin{figure*}

\begin{mathpar}
\inferrule*[right=(SKIP)]
{\hspace{1mm}}
{\sigma,\skp;C \bpstep \sigma,C}

\inferrule*[right=(ASSGN)]
{\hspace{1mm}}
{(s,h,f),x:=E \bpstep (s[x \mapsto \den{E}s],h,f),\skp}

\inferrule*[right=(HEAP-READ)]
{\den{E}s \in \dom{h}}
{(s,h,f),x:=[E] \bpstep (s[x \mapsto h(\den{E}s)],h,f),\skp}

\inferrule*[right=(HEAP-WRITE)]
{\den{E}s \in \dom{h}}
{(s,h,f),[E]:=E' \bpstep (s,h[\den{E}s \mapsto \den{E'}s],f),\skp}

\inferrule*[right=(CONS)]
{\forall i \in [1,n] \such l+i-1 \in f}
{(s,h,f),x:=\cons(E_1,\ldots,E_n) \bpstep \\
(s[x \mapsto l], h[l \mapsto \den{E_1}s]\ldots[l+n-1 \mapsto \den{E_n}s], f-\{l,\ldots,l+n-1\}),\skp}

\inferrule*[right=(FREE)]
{\den{E}s \in \dom{h}}
{(s,h,f),\free(E) \bpstep (s,h \backslash \den{E}s,f \cup \{\den{E}s\}),\skp}

\inferrule*[right=(SEQ)]
{\sigma,C \bpstep \sigma',C'}
{\sigma,C;C'' \bpstep \sigma',C';C''}

\inferrule*[right=(IF-TRUE)]
{\den{B}s = \texttt{true}}
{\sigma,\condfull{B}{C_1}{C_2} \bpstep \sigma,C_1}

\inferrule*[right=(IF-FALSE)]
{\den{B}s = \texttt{false}}
{\sigma,\condfull{B}{C_1}{C_2} \bpstep \sigma,C_2}

\inferrule*[right=(WHILE-TRUE)]
{\den{B}s = \texttt{true}}
{\sigma,\while{B}{C} \bpstep \sigma,C;\while{B}{C}}

\inferrule*[right=(WHILE-FALSE)]
{\den{B}s = \texttt{false}}
{\sigma,\while{B}{C} \bpstep \sigma,\skp}
\end{mathpar}

\caption{Small-Step Operational Semantics}
\label{opsem}
\end{figure*}

\else

The small-step operational semantics for our machine is defined as
$\sigma,C \bpstep \sigma',C'$ and is straightforward; the full details can
be found in the extended TR. The most interesting aspects are the rules 
for allocation and deallocation, since they make use of the free list.
$x:=\cons(E_1,\ldots,E_n)$ allocates a nondeterministically-chosen
contiguous block of $n$ heap cells from the free list, while $\free(E)$
puts the single heap cell pointed to by $E$ back onto the free list.
None of the operations make use of any memory existing outside the program 
state~--- this is the key for obtaining behavior-preservation.
\fi

To see how out state model fits into the structure defined in Section~\ref{locality},
we need to define the state combination operator. Given two states $\sigma_1 = (s_1,h_1,f_1)$ and
$\sigma_2 = (s_2,h_2,f_2)$, the combined state $\sigma_1 \dt \sigma_2$ is equal to 
$(s_1,h_1 \uplus h_2,f_1)$ if $s_1 = s_2$, $f_1 = f_2$, and the domains of $h_1$ and $h_2$ are
disjoint; otherwise, the combination is undefined. Note that this combined state satisfies the
requisite condition $\text{dom}(h_1 \uplus h_2) \cap f_1 = \emp$ because $h_1$, $h_2$, and $f_1$
are pairwise disjoint by assumption. The most important aspect of this definition of state 
combination is that we can never change the free list when adding extra resources. This guarantees
behavior preservation of the nondeterministic memory allocator because the allocator's set of 
possible behaviors is precisely defined by the free list.

In order to formally compare our logic to standard Separation Logic, we 
need to provide the standard version of the small-step operational semantics,
denoted as $(s,h),C \istep (s',h'),C'$. This definition is nearly identical
to Figure~\ref{opsem}, except that all free lists are removed from program
state, and the (CONS) rule precondition is modified to only require that
the newly-allocated locations are not in $\dom{h}$.
\ifextended
It is then possible to show the following relationship between the
two operational semantics (note that the full proofs for all of the following 
lemmas and theorems can be found in our Coq implementation~\cite{costanzo-thesis}):

\begin{lem}
\label{opequiv}
\begin{align*}
(s,h),C \istepn{n} & (s',h'),C' \iff \exists f,f' \such (s,h,f),C \bpstepn{n} (s',h',f'),C'
\end{align*}
\end{lem}
\else
We formalize this semantics in the extended TR, and prove the following relationship between the
two operational semantics:
{\small
\begin{align*}
(s,h),C \istepn{n} & (s',h'),C' \iff \exists f,f' \such (s,h,f),C \bpstepn{n} (s',h',f'),C'
\end{align*}}
\fi
\ifextended
\begin{proof}
The backwards direction is a straightforward proof by induction. For the forwards direction, we actually
prove a stronger statement by picking our $f$ and $f'$ to be exactly $\N-\dom{h}$ and $\N-\dom{h'}$,
respectively. The proof of this stronger statement is then straightforward by induction. Picking the free
lists in this way showcases how the Separation Logic model can be interpreted as having an implicit free list
containing everything not in the heap.
\end{proof}

\fi
\ifextended
The inference rules in the form $\assertd{P}{C}{Q}$ for our logic
are exactly the same as those used in standard Separation Logic.
We give many of these inference rules in Figure~\ref{infrules}; the
reader may refer to \cite{reynolds02} for more.
%%
%%%%%%%%%%%%%%%%%%%%%%%%%%%%%%%%%%%%%%%%%%%%%%%%%%%%%%%%%%%%%%%%%%%%%%%%%%%%
\begin{figure}
\begin{mathpar}
\inferrule*[right=(SKIP)]
{\hspace{1mm}}
{\assertd{\empa}{\skp}{\empa}}
%
~~~
\inferrule*[right=(ASSGN)]
{\hspace{1mm}}
{\assertd{x=y \land \empa}{x:=E}{x=E\sub{y}{x} \land \empa}}

\inferrule*[right=(HEAP-READ)]
{\hspace{1mm}}
{\assertd{x=y \land E \mapsto z}{x:=[E]}{x=z \land E\sub{y}{x} \mapsto z}}

\inferrule*[right=(HEAP-WRITE)]
{\hspace{1mm}}
{\assertd{E \mapsto -}{[E]:=E'}{E \mapsto E'}}

\inferrule*[right=(CONS)]
{\hspace{1mm}}
{\assertd{x=y \land \empa}{x:=\cons(E_1,\ldots,E_k)}{x \mapsto E_1\sub{y}{x},\ldots,E_k\sub{y}{x}}}

\inferrule*[right=(FREE)]
{\hspace{1mm}}
{\assertd{E \mapsto -}{\free(E)}{\empa}}

\inferrule*[right=(SEQ)]
{\assertd{P}{C_1}{Q} \\
\assertd{Q}{C_2}{R}}
{\assertd{P}{C_1;C_2}{R}}

\inferrule*[right=(IF)]
{\assertd{B \land P}{C_1}{Q} \\
\assertd{\lnot B \land P}{C_2}{Q}}
{\assertd{P}{\condfull{B}{C_1}{C_2}}{Q}}

\inferrule*[right=(WHILE)]
{\assertd{B \land P}{C}{P}}
{\assertd{P}{\while{B}{C}}{\lnot B \land P}}

\inferrule*[right=(CONSEQ)]
{P' \Rightarrow P \\
Q \Rightarrow Q' \\
\assertd{P}{C}{Q}}
{\assertd{P'}{C}{Q'}}

\inferrule*[right=(CONJ)]
{\assertd{P_1}{C}{Q_1} \\
\assertd{P_2}{C}{Q_2}}
{\assertd{P_1 \land P_2}{C}{Q_1 \land Q_2}}

\inferrule*[right=(DISJ)]
{\assertd{P_1}{C}{Q_1} \\
\assertd{P_2}{C}{Q_2}}
{\assertd{P_1 \lor P_2}{C}{Q_1 \lor Q_2}}

\inferrule*[right=(FRAME)]
{\assertd{P}{C}{Q} \\
\texttt{modifies}(C) \cap \texttt{vars}(R) = \emp}
{\assertd{P*R}{C}{Q*R}}
\end{mathpar}
%%%%%%%%%%%%%%%%%%%%%%%%%%%%%%%%%%%%%%%%%%%%%%%%%%%%%%%%%%%%%%%%%%%%%%%%%%%%
\caption{Some Separation Logic Inference Rules}
\label{infrules}
\end{figure}
%%

\fi
\ifextended
We next define safe execution and semantic triples. A configuration $(\sigma,C)$
is \emph{safe} if it can never get stuck in a non-halting state:
\begin{align*}
\bpsafe(\sigma,C) \isdef \, \forall \sigma',C' \such 
\sigma,C \bpsteps \sigma',C' \, \land \, C' \neq \skp 
\,\, \Longrightarrow \,\, \exists \sigma'',C'' \such \sigma',C' \bpstep \sigma'',C''
\end{align*}
\noindent{}
A triple $\assertm{P}{C}{Q}$ is then \emph{semantically valid} when, for all $\sigma$, $\sigma'$:
\begin{align*}
& \text{1.) if } \sigma \models P\text{, then } \bpsafe(\sigma,C) \\
& \text{2.) if } \sigma \models P \text{ and } \sigma,C \bpsteps \sigma',\skp \text{, then } \sigma' \models Q
\end{align*}
Semantic validity of standard Separation Logic triples is defined in the same way, but using the operational semantics
for Separation Logic. We will write this as $\assertsl{P}{C}{Q}$. Note that we are only considering a
\emph{partial correctness} definition of validity here, meaning that programs are not required to terminate.

We now formally relate semantic validity of our logic with standard Separation Logic,
with the help of a minor technical lemma:

\begin{lem}
\label{safestep}
\begin{align*}
(s,h), & C \istep (s',h'),C' \Longrightarrow \forall f \such (f \# h \Rightarrow \exists \sigma \such (s,h,f),C \bpstep \sigma,C')
\end{align*}
\end{lem}

\begin{proof}
Straightforward by induction on the rules for stepping.
\end{proof}

\begin{thm}[Equivalence of Semantic Validity]
\label{semantic-equiv}
\begin{align*}
\assertsl{P}{C}{Q} \iff \assertm{P}{C}{Q}
\end{align*}
\end{thm}

\begin{proof}
First, suppose that $\assertsl{P}{C}{Q}$. To prove the first property of semantic validity, suppose that
$(s,h,f) \models P$, and consider some execution $(s,h,f),C \bpsteps (s',h',f'),C'$ with
$C' \neq \skp$. Then we need to show that $(s',h',f'),C'$ can take another step. By Lemma~\ref{opequiv},
we have that $(s,h),C \isteps (s',h'),C'$. Since $(s,h) \models P$, we know that $\bpsafe((s,h),C)$, and
so $(s',h'),C' \istep (s'',h''),C''$ for some $s''$, $h''$, $C''$. Therefore Lemma~\ref{safestep} tells
us that $(s',h',f'),C'$ can indeed take a step. For the second property, suppose that $(s,h,f) \models P$
and $(s,h,f),C \bpsteps (s',h',f'),\skp$. Then Lemma~\ref{opequiv} tells us that $(s,h),C \isteps (s',h'),\skp$,
meaning that $(s',h') \models Q$, and so $(s',h',f') \models Q$.

Now suppose that $\assertm{P}{C}{Q}$. For the first property, suppose that $(s,h) \models P$ and 
$(s,h),C \isteps (s',h'),C'$ with $C' \neq \skp$. Lemma~\ref{opequiv} gives us 
$(s,h,f),C \bpsteps (s',h',f'),C'$ for some $f$ and $f'$, which means that $(s',h',f'),C' \bpstep (s'',h'',f''),C''$
for some $s''$, $h''$, $f''$, $C''$ (since $(s,h,f) \models P)$. Therefore Lemma~\ref{opequiv} 
gives us $(s',h'),C' \istep (s'',h''),C''$, as desired. For the second property, suppose $(s,h) \models P$
and $(s,h),C \isteps (s',h'),\skp$. By Lemma~\ref{opequiv}, we have 
\[
(s,h,f),C \isteps (s',h',f'),\skp
\]
for some $f$ and $f'$. Since $(s,h,f) \models P$, this means that $(s',h',f') \models Q$, and so
$(s',h') \models Q$. 
\end{proof}

\begin{thm}[Soundness and Completeness]
\label{sound-complete}
\begin{align*}
\assertd{P}{C}{Q} \iff \assertm{P}{C}{Q}
\end{align*}
\end{thm}

\begin{proof}
Note that $\assertd{P}{C}{Q}$ has the same definition in both our logic and in Separation Logic, since we
use the same assertion language and inference rules. Therefore, because Separation Logic is known to be
sound and complete, we have that $\assertd{P}{C}{Q} \iff \assertsl{P}{C}{Q}$. Applying
Theorem~\ref{semantic-equiv} gives the desired result.
\end{proof}

\fi
\ifextended

We have thus shown that our new model does not cause any complications in the usage of Separation Logic.
\else
\noindent{}The inference rules in the form $\assertd{P}{C}{Q}$ for our logic
are same as those used in standard Separation Logic.
In the extended TR, we state all the inference rules and prove that our logic is both sound and 
complete; therefore, behavior preservation does not cause any complications in the usage of Separation 
Logic. 
\fi
Any specification that can be proved using the standard model can also be proved using our model,
with the exact same application of inference rules (since they completely ignore the
free list within program state). 
\ifextended
We now
only need to show that our model enjoys the stronger, behavior-preserving notion of locality. As
described in Section~\ref{locality}, this locality is composed of Safety Monotonicity, Termination
Equivalence, and the Forward and Backwards Frame Properties. We first prove that the two frame 
properties hold:

\begin{thm}[Frame Properties]
\begin{align*}
& \text{1.) } (s,h_0,f),C \bpstepn{n} (s',h_0',f'),C' \land h_0 \# h_1 \land f \# h_1 \Longrightarrow \\
& \hspace{8mm} h_0' \# h_1 \land (s,h_0 \dt h_1,f),C \bpstepn{n} (s',h_0' \dt h_1,f'),C' \\
& \text{2.) } \bpsafe((s,h_0,f),C) \land (s,h_0 \dt h_1,f),C \bpstepn{n} (s',h',f'),C' \Longrightarrow \\
& \hspace{8mm} \exists h_0' \such h' = h_0' \dt h_1 \land (s,h_0,f),C \bpstepn{n} (s',h_0',f'),C'
\end{align*}
\end{thm}

\begin{proof}
Straightforward by induction on the derivation rules for stepping. For details, see the Coq implementation.
\end{proof}

It is then easy to show that these Frame Properties imply both Safety Monotonicity and Termination Equivalence.

\begin{lem}[Safety Monotonicity]
\begin{align*}
\bpsafe & ((s,h_0,f),C) \land h_0 \# h_1 \land f \# h_1 \Longrightarrow \bpsafe((s,h_0 \dt h_1,f),C)
\end{align*}
\end{lem}

\begin{proof}
Suppose that $\bpsafe((s,h_0,f),C)$, and consider an execution on the large state 
$(s,h_0 \dt h_1,f),C \bpstepn{n} (s',h',f'),C'$ with $C' \neq \skp$. Then the Backwards Frame Property tells us
that $h' = h_0' \dt h_1$ and $(s,h_0,f),C \bpstepn{n} (s',h_0',f'),C'$. Since $\bpsafe((s,h_0,f),C)$ and $C' \neq \skp$,
we see that $(s',h_0',f'),C' \bpstep (s'',h_0'',f''),C''$ for some $s''$, $h_0''$, $f''$, $C''$. Thus we can now use the 
Forwards Frame Property (clearly $h_1 \# f'$ since $(s',h_0' \dt h_1,f')$ is a well-typed state) to obtain 
$(s',h_0' \dt h_1,f'),C' \bpstep (s'',h_0'' \dt h_1,f''),C''$, and so $\bpsafe((s,h_0 \dt h_1,f),C)$ does indeed hold.
\end{proof}

In order to define 
Termination Equivalence, we first need to define divergence. We say that $\sigma$ diverges on $C$,
written  $\sigma,C \diverges$, if there exists an infinite path of steps starting from $\sigma,C$. More formally:
\begin{align*}
\sigma,C \diverges & \isdef \forall n \such \exists \sigma',C' \such \sigma,C \bpstepn{n} \sigma',C'
\end{align*}

\begin{lem}[Termination Equivalence]
\begin{align*}
\bpsafe & ((s,h_0,f),C) \land h_0 \# h_1 \land f \# h_1 \Rightarrow ((s,h_0,f),C \diverges \iff (s,h_0 \dt h_1,f),C \diverges)
\end{align*}
\end{lem}

\begin{proof}
First, suppose $(s,h_0,f),C \diverges$, and pick any $n$. Then there are some $s'$, $h_0'$, $f'$, $C'$ such
that $(s,h_0,f),C \bpstepn{n} (s',h_0',f'),C'$. Thus the Forwards Frame Property tells us that $h_0' \# h_1$ and
$(s,h_0 \dt h_1,f),C \bpstepn{n} (s',h_0' \dt h_1,f'),C'$, as desired. For the other direction, suppose
$(s,h_0 \dt h_1,f),C$ and pick any $n$. Then  $(s,h_0 \dt h_1,f),C \bpstepn{n} (s',h',f'),C'$ for some $s'$,
$h'$, $f'$, $C'$. Since $\bpsafe((s,h_0,f),C)$, the Backwards Frame Property tells us 
that $h' = h_0' \dt h_1$ for some $h_0'$, and $(s,h_0,f),C \bpstepn{n} (s',h_0',f'),C'$, as desired.
\end{proof}

\else
Also in the TR, we prove that 
our model enjoys the stronger, behavior-preserving notion of locality
described in Sec~\ref{locality}.
\fi

\begin{comment}
\ifextended
We conclude this section with a quick note on reasoning about the free list. We presented our logic with
the purpose of showing that, at the level of inference rules and derivations, it works exactly the same as
standard Separation Logic. However, at the level of the underlying model, we now have this free list within the state.
Therefore, if we so desire, we could define additional assertions and inference rules allowing for more precise reasoning
involving the free list. One idea might be to have a separate, free list section of assertions in which we write,
for example, $E*\true$ to claim that $E$ is a part of the free list. Then the axiom for $\free$
would look something like:
{\small
\begin{align*}
\assert{E \mapsto -; \true}{\free(E)}{\empa; E*\true}
\end{align*}}

\else
Even though our logic works exactly the same as
standard Separation Logic, our underlying
model now has this free list within the state.  Therefore, if we
so desire, we could define additional assertions and inference rules
allowing for more precise reasoning involving the free list. One idea is to have
a separate, free list section of assertions in which we write, for
example, $E*\true$ to claim that $E$ is a part of the free list. Then
the axiom for $\free$ would look like:
\begin{align*}
\assert{E \mapsto -; \true}{\free(E)}{\empa; E*\true}
\end{align*}

\fi
\end{comment}
