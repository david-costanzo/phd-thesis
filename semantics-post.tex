\label{logic-chapter}
Our work published in POST 2014~\cite{costanzo-ddifc}
presents a program logic for proving that code is secure with respect
to a general, expressive security policy. This chapter will present the formalization of this
program logic, and will show how a particular nontrivial program is verified secure
using the logic. However, after considering the implications of using
the logic, as well as its relative incompleteness, we will arrive at
a crucial conclusion: it would be far more ideal to verify security
in a way that is not specific to a particular program logic. Later
chapters then demonstrate that this ideal is in fact achievable.
Hence this chapter should be taken with a grain of salt; it is important
for developing a solid understanding of formal security verification
(especially with respect to policy specification), but this
dissertation ultimately advocates \emph{against} relying upon any specific program 
logic such as the one presented here.

\section{Program Logic Overview}

The ultimate goal of our security-aware program logic is to verify
interesting security policies over low-level systems code. For convenience
we will use a toy C-like language here that supports heap manipulation and
pointer arithmetic; it should generally be clear how to
extend the toy language with other C features. The language will not support
unstructured control flow (e.g., ``goto''), however, so it is not designed
for directly reasoning about assembly code. This is one reason why we will
eventually decide to move away from a specific program logic and instead
design a more general methodology that works for any code written in any
language.

Our program logic will perform security reasoning by directly modeling 
dynamic label tainting (as described in Section~\ref{intro-policies}).
For simplicity, we will use a label lattice consisting only of the two
labels \lo{} and \hi{}, with $\lo{} \sqsubseteq \hi{}$. This lattice
can easily be mapped to a more intricate one, however: for a given observer
principal $p$ with label $L_p$, the \lo{} label represents all labels in the 
more intricate lattice which are equal to or below $L_p$, while \hi{} represents
all other labels.

Label propagation is done in a mostly obvious way. All values in the variable
store and the global memory heap are associated with a label (\lo{} or \hi{}).
If we have a direct assignment such as $x:=y$,
then the label of $y$'s data propagates into $x$ along with the data itself. We compute the composite
label of an expression such as $2*x + z$ to be the least upper bound of the labels of its constituent
parts (for the two-element lattice of \lo{} and \hi{}, this will be \lo{} if and only if each 
constituent label is \lo{}). For the heap-read command $x:=[E]$, we must propagate both the 
label of $E$ and the label of the data located at heap address $E$ into $x$. In other words, if
we read some low-security data from the heap using a high-security pointer, the result must
be tainted as high security in order for our information flow tracking to be accurate.
Similarly, the heap-write command $[E]:=E'$ must propagate both the label of $E'$ and the label of 
pointer $E$ into the location $E$ in the heap. As a general rule for any of these atomic commands,
we compute the composite label of the entire read-set, and propagate that into all locations in
the write-set.

\subsection{Security Formulation}

Our security guarantee is based on pure noninterference,
which says that the initial values of \hi{} data have no effect on the ``observable behavior'' of
a program's execution. We choose to define observable behavior in terms of a special output channel.
We include an output command in our language, and an execution's observable behavior is defined to
be exactly the sequence of values that the execution outputs.

The standard way to express this noninterference property formally is in terms of two executions:
a program is deemed to be noninterfering if two executions of the program from \emph{observably
equivalent} initial states always yield identical outputs. Two states are defined to be observably
equivalent when only their high-security values differ. Thus this property describes what one would 
expect: changing the value of any high-security data in the initial state will cause no change in 
the program's output. As hinted at during the examples of Section~\ref{intro-policies}, we will 
actually use a weaker version of noninterference to allow for security policies with certain 
well-specified flows from high security to low (e.g., declassification).

We accomplish this weakening of noninterference by
requiring a precondition to hold on the initial state of an execution. That is, we alter the 
noninterference property to say that two executions will yield identical outputs if they start from 
two observably equivalent states that both satisfy some state predicate $P$. This weakening 
is interesting for two reasons. First, it provides a clean interface between information-flow 
reasoning and Hoare Logic (a program logic that derives pre/postconditions as state predicates). 
Second, this weaker property describes a certain level of dependency between high-security inputs 
and low-security outputs, rather than the complete independence of pure noninterference. This is
key for allowing interesting, real-world security policies.
To better understand this property, let us revisit the examples of Section~\ref{intro-policies}.

\paragraph{Public Parity}
Suppose we have a variable $x$ that contains Alice's secret value $v$, and Alice wishes
to only publicly release the parity of $v$. We can prove that some program $C$ obeys this
security policy by verifying it using our program logic, with respect to a 
precondition $P$ saying ``$x$ contains high data, $y$ contains low data, and $y = x \% 2$''.
Our weakened noninterference property described above says that if we have an execution from 
some initial state satisfying this $P$,
then changing the value of $x$ will not affect the program's output as long as the new state also 
satisfies $P$. Since $y$'s value is $v\%2$ and is unchanged in the two executions, this means that 
as long as we change $x$'s value to something \emph{with the same parity}, the output will be
unchanged. Indeed, this is exactly the desired property for enforcing Alice's security policy.
In this way, the precondition $P$ used for verifying $C$ serves as a formal description of
the security policy.

\paragraph{Public Average}
Suppose we have $n$ salaries stored in variables $x_1$ through $x_n$, and we are only willing to 
release their average as public. This 
is similar to the previous example, except that we now have multiple secrets. The precondition 
$P$ says that $x_1$ through $x_n$ all contain \hi{} data, $a$ contains \lo{} data, and 
$a = (\Sigma~x_i)/n$. 
In this situation, noninterference of a verified program $C$ says that we can change any of 
the values contained within the \emph{set} 
of variables $\{x_1,...,x_n\}$ in any way, and $C$'s output will be unaffected as long as the 
average of all the values is unchanged.

\paragraph{Shared Calendar}
Suppose Alice has a calendar represented as an array of $n$ time slots at heap locations $l$ 
through $l+n-1$. Each time slot may either contain a value of $0$ if Alice has not scheduled an
event at that time, or some nonzero value encoding the details of an event. The 
precondition $P$ says that, for each time slot $t$ in the calendar, either the value at $t$
is $0$ and the label is \lo{}, or the value is not $0$ and the label is \hi{}. This is an 
example of what we call a \emph{conditional label}: a label that is dependent on values
within the program state (notice that the previous examples did not use conditional labels).
Interpreting this precondition as our security policy, noninterference of a program $C$ says 
that changing any \emph{nonzero} (i.e., high security) values within the calendar will not 
have any effect on the output produced by $C$ as long as $P$ still holds; in order for $P$ to 
still hold, we must be changing these values into other \emph{nonzero} values. Hence we
get exactly the desired policy: the observable behavior is independent from the details
of any scheduled events (i.e., we cannot distinguish one nonzero value from another, but
we \emph{can} determine which time slots contain $0$ and which do not). 
Conditional labels are a novelty of our program logic; they allow for highly expressive
security policies, and we will show below
how they can be a useful and powerful tool in conducting the security verification.

\paragraph{Dynamic Label Tainting}
We can trivially represent standard dynamic label tainting, since our program logic
is based upon directly modeling labels and their propagation throughout an execution.
It is important to note, however, that this modeling of labels is purely logical.
The labels are there to help specify the security policy and to help conduct the security
verification; when the actual program executes, no labels exist within the machine
state, and no computations occur for propagating labels. We enforce this formally
by defining two different machine operational semantics. The low-level, ``true'' semantics 
is completely unaware of security labels, while the higher level instrumented semantics
deals with labels and propagation. We prove standard theorems to relate the two semantics
together, so that our final end-to-end security guarantee applies to the true semantics
of execution.

\section{Language and Semantics}
\label{semantics}

We can now get into the formal definitions of our program logic.
The entire program logic, along with its theorems, have been encoded
in the Coq proof assistant language and can be found at the
companion website~\cite{costanzo-thesis}.
Our simple, C-like toy programming language is defined as follows: 
\begin{mathpar}

\begin{bnf}[r@{\ \ \ }c@{\ }]

\production{\text{(Exp)} & E}
    \prodcase{x}
    \prodcase{c}
    \prodcase{E + E}
    \prodcase{\cdots}
    
\production{\text{(BExp)} & B}
    \prodcase{\ttt{false}}
    \prodcase{E=E}
    \prodcase{B \land B}
    \prodcase{\cdots}

\production{\text{(Cmd)} & C}
    \prodcase{\skp}
    \prodcase{\out{E}}
    \prodcase{x:=E}
    \prodcase{x:=[E]}
    \prodcase{[E]:=E}
    \prodcase{\seq{C}{C}}
    \prodcase{\condfull{B}{C}{C}}
    \prodcase{\while{B}{C}}
    
\end{bnf}

\end{mathpar}
Valid code includes variable assignment, heap load/store, if statements, while loops, and output.
Our model of a program state, consisting of a variable store and a heap, is given by:
\begin{mathpar}

\begin{bnf}[r@{\ \ \ }c@{\ }]
    
\production{(\text{Lbl}) & L}
    \prodcase{\lo{}}
    \prodcase{\hi{}}    
    
\production{(\text{Val}) & V}
    \prodcase{\mathbb{Z} \times \text{Lbl}}

\production{(\text{Store}) & s}
    \prodcase{\text{Var} \rightarrow \text{option } \text{Val}}
    
\production{(\text{Heap}) & h}
    \prodcase{\mathbb{N} \rightarrow \text{option } \text{Val}}

\production{(\text{State}) & \sigma}
    \prodcase{\text{Store} \times \text{Heap}}
    
\end{bnf}

\end{mathpar}
\begin{comment}
The denotational semantics of expressions is defined as:
\begin{align*}
\den{E} \,\, & : \,\, \text{Store} \rightarrow \text{option Val} \\
\den{x}s & = s(x) \\
\den{c}s & = \some{(c,\lo)} \\
\den{E_1 \ttt{ op } E_2}s & = \left\{
\begin{aligned}
& \some{(v_1 \den{\ttt{op}} v_2, l_1 \sqcup l_2)}, \quad \text{if } \\
& \qquad \left(\begin{aligned}
& \den{E_1}s = \some{(v_1,l_1)} \\
& \text{ and } \, \den{E_2}s = \some{(v_2,l_2)} \\
\end{aligned}\right. \\
& \none, \qquad \text{otherwise}
\end{aligned}\right.
\end{align*}
\end{comment}
Given a variable store $s$, we define a denotational semantics $\den{E}s$ that evaluates an expression to a pair of 
integer and label, with the label being the least upper bound of the labels of the constituent parts. The denotation 
of an expression also may evaluate to \none{}, indicating that the program state does not contain the necessary 
resources to evaluate. We have a similar denotational semantics for boolean expressions. The formal definitions of 
these semantics are omitted here as they are standard and straightforward. Note that we will sometimes write
 $\den{E}\sigma$ as shorthand for $\den{E}$ applied to the store of state $\sigma$.


%%%%%%%%%%%%%%%%%%%%%%%%%%%%%%%%%%%%%%%%%%%%%%%%%%%%%%%%%%%%%%%%%%%%%%%%%%%%%%%%%%%%%%%%%%%%%%%%%%%%%%%%%%%%%%%%%%%%%%%
\begin{figure}[!p]

\small{
\begin{mathpar}
\inferrule*[right=(ASSGN)]
{\den{E}s = \some{(n,l)}}
{\pconfig{(s,h)}{x:=E}{K} \pstep{l'}{} \pconfig{(s[x \mapsto (n, l \sqcup l')],h)}{\skp}{K}}

\inferrule*[right=(READ)]
{\den{E}s = \some{(n_1,l_1)} \\
h(n_1) = \some{(n_2,l_2)}}
{\pconfig{(s,h)}{x:=[E]}{K} \pstep{l'}{} \pconfig{(s[x \mapsto (n_2, l_1 \sqcup l_2 \sqcup l')],h)}{\skp}{K}}

\inferrule*[right=(WRITE)]
{\den{E}s = \some{(n_1,l_1)} \\
h(n_1) \neq \none{} \\
\den{E'}s = \some{(n_2,l_2)}}
{\pconfig{(s,h)}{[E]:=E'}{K} \pstep{l'}{} \pconfig{(s,h[n_1 \mapsto (n_2, l_1 \sqcup l_2 \sqcup l')])}{\skp}{K}}

\inferrule*[right=(OUTPUT)]
{\den{E}\sigma = \some{(n,\lo)}}
{\pconfig{\sigma}{\out{E}}{K} \pstep{\lo}{[n]} \pconfig{\sigma}{\skp}{K}}

\inferrule*[right=(IF-TRUE)]
{\den{B}\sigma = \some{(\true,l)} \\
l \sqsubseteq l'}
{\pconfig{\sigma}{\condfull{B}{C_1}{C_2}}{K} \pstep{l'}{} \pconfig{\sigma}{C_1}{K}}

\inferrule*[right=(IF-FALSE)]
{\den{B}\sigma = \some{(\false,l)} \\
l \sqsubseteq l'}
{\pconfig{\sigma}{\condfull{B}{C_1}{C_2}}{K} \pstep{l'}{} \pconfig{\sigma}{C_2}{K}}

\inferrule*[right=(IF-HI)]
{\den{B}\sigma = \some{(\_,\hi)} \\
\pconfig{\ttt{mark\_vars}(\sigma,\condfull{B}{C_1}{C_2})}{\condfull{B}{C_1}{C_2}}{[]} \pstepn{\hi}{n}{} \pconfig{\sigma'}{\skp}{[]}}
{\pconfig{\sigma}{\condfull{B}{C_1}{C_2}}{K} \pstep{\lo}{} \pconfig{\sigma'}{\skp}{K}}

%\inferrule*[right=(IF-HI-DVG)]
%{\den{B}s = (\_,\hi) \\
%\forall \sigma' \such \pconfig{\sigma}{\condfull{B}{C_1}{C_2}}{[]} \psteps{\hi}{}\!\!\!\!\!\!\!\!\!\!\not \quad\;\;  \pconfig{\sigma'}{\skp}{[]}}
%{\pconfig{\sigma}{\condfull{B}{C_1}{C_2}}{K} \pstep{\lo}{} \pconfig{\sigma}{\condfull{B}{C_1}{C_2}}{K}}

\inferrule*[right=(WHILE-TRUE)]
{\den{B}\sigma = \some{(\true,l)} \\
l \sqsubseteq l'}
{\pconfig{\sigma}{\while{B}{C}}{K} \pstep{l'}{} \pconfig{\sigma}{C; \while{B}{C}}{K}}

\inferrule*[right=(WHILE-FALSE)]
{\den{B}\sigma = \some{(\false,l)} \\
l \sqsubseteq l'}
{\pconfig{\sigma}{\while{B}{C}}{K} \pstep{l'}{} \pconfig{\sigma}{\skp}{K}}

\inferrule*[right=(WHILE-HI)]
{\den{B}\sigma = \some{(\_,\hi)} \\
\pconfig{\ttt{mark\_vars}(\sigma,\while{B}{C})}{\while{B}{C}}{[]} \pstepn{\hi}{n}{} \pconfig{\sigma'}{\skp}{[]}}
{\pconfig{\sigma}{\while{B}{C}}{K} \pstep{\lo}{} \pconfig{\sigma'}{\skp}{K}}

%\inferrule*[right=(WHILE-HI-DVG)]
%{\den{B}s = (\_,\hi) \\
%\forall \sigma' \such \pconfig{\sigma}{\while{B}{C}}{[]} \psteps{\hi}{}\!\!\!\!\!\!\!\!\!\!\not \quad\;\;  \pconfig{\sigma'}{\skp}{[]}}
%{\pconfig{\sigma}{\while{B}{C}}{K} \pstep{\lo}{} \pconfig{\sigma}{\while{B}{C}}{K}}

\inferrule*[right=(SEQ)]
{ }
{\pconfig{\sigma}{\seq{C_1}{C_2}}{K} \pstep{l}{} \pconfig{\sigma}{C_1}{C_2 :: K}}

\inferrule*[right=(SKIP)]
{ }
{\pconfig{\sigma}{\skp}{C::K} \pstep{l}{} \pconfig{\sigma}{C}{K}}

\inferrule*[right=(ZERO)]
{ }
{\pconfig{\sigma}{C}{K} \pstepn{l}{0}{} \pconfig{\sigma}{C}{K}}

\inferrule*[right=(SUCC)]
{\pconfig{\sigma}{C}{K} \pstep{l}{o} \pconfig{\sigma'}{C'}{K'} \\
\pconfig{\sigma'}{C'}{K'} \pstepn{l}{n}{o'} \pconfig{\sigma''}{C''}{K''} \\
n > 0}
{\pconfig{\sigma}{C}{K} \pstepn{l}{n+1}{o ++ o'} \pconfig{\sigma''}{C''}{K''}}
\end{mathpar}
}
\vspace{-30pt}
\caption{Security-Aware Operational Semantics}
\label{sem}
\end{figure}
%%%%%%%%%%%%%%%%%%%%%%%%%%%%%%%%%%%%%%%%%%%%%%%%%%%%%%%%%%%%%%%%%%%%%%%%%%%%%%%%%%%%%%%%%%%%%%%%%%%%%%%%%%%%%%%%%%%%%%%

Figure~\ref{sem} defines our instrumented operational semantics. The semantics is 
security-aware, meaning that it keeps track of security
labels on data and propagates these labels throughout execution in order to track which 
values might have been influenced by
some high-security data. The semantics operates on machine configurations, which consist 
of program state, code, and a list of 
commands called the continuation stack (we use a continuation-stack approach solely for 
the purpose of simplifying some 
proofs). The transition arrow of the semantics is annotated with a \emph{program counter label}, 
which is a standard IFC 
construct used to keep track of information flow resulting from the control flow of the 
execution. Whenever an execution 
enters a conditional construct, it raises the pc label by the label of the boolean 
expression evaluated; the pc label then taints any assignments made 
within the conditional construct (see variable $l'$ in the (ASSGN), (READ), and (WRITE) 
rules). The transition arrow is also annotated with a
list of outputs (equal to the empty list when not explicitly written) and the number of 
steps (equal to 1 when not explicitly written).
We sometimes annotate the arrow with an asterisk ($\psteps{}{}$) to indicate zero
or more steps.

Two of the rules for conditional constructs make use of a function 
called \ttt{mark\_vars}. $\ttt{mark\_vars}(\sigma, C)$ alters $\sigma$ by setting the label 
of each variable in $\ttt{modifies}(C)$ to \hi{}, where $\ttt{modifies}(C)$ is a 
syntactic function returning an overapproximation of the store variables that may be modified 
by $C$.
\begin{comment}
\begin{align*}
\ttt{modifies}(\skp) & = \emptyset \\
\ttt{modifies}(\out{E}) & = \emptyset \\
\ttt{modifies}(x:=E) & = \{x\} \\
\ttt{modifies}(x:=[E]) & = \{x\} \\
\ttt{modifies}([E]:=E') & = \emptyset \\
\ttt{modifies}(\seq{C_1}{C_2}) & = \ttt{modifies}(C_1) \cup \ttt{modifies}(C_2) \\
\ttt{modifies}(\condfull{B}{C_1}{C_2}) & = \ttt{modifies}(C_1) \cup \ttt{modifies}(C_2) \\
\ttt{modifies}(\while{B}{C}) & = \ttt{modifies}(C)
\end{align*}
\end{comment}
Thus, whenever we raise the pc label to \hi{}, our semantics taints all store variables 
that appear on the left-hand side of an assignment or heap-read command within the 
conditional construct, even if some of those commands do not actually get executed. Note 
that regardless of which branch of an if statement is taken, the semantics taints all the 
variables in \emph{both} branches. This is required for noninterference, due to the 
well-known fact that the \emph{lack} of assignment in a branch of an if statement can 
leak information about the branching expression. Consider, for example, the following program:
\begin{alltt}
1    y := 1;
2    if (x = 0) then y := 0 else skip;
3    if (y = 0) then skip else output 1;
\end{alltt} 
Suppose $x$ contains \hi{} data initially, while $y$ contains \lo{} data.
If $x$ is 0, then $y$ will be assigned 0 at line 2 and tainted with a \hi{} label (by the pc label).
Then nothing happens at line 3, and the program produces no output. If $x$ is nonzero, however, nothing
happens at line 2, so $y$ still has a \lo{} label at line 3. Thus the output command at line 3 executes
without issue. Therefore the output of this program depends on the \hi{} data in $x$, even though our
instrumented semantics executes safely. We choose to resolve this issue by using the \ttt{mark\_vars}
function in the semantics. Then $y$ will be tainted at line 2 regardless of the value of $x$, and so the
semantics will get stuck at line 3 when $x$ is nonzero. In other words, we would only be able to verify this
program with a precondition saying that $x = 0$~--- the program is indeed noninterfering
with respect to this degenerate security policy.

%%%%%%%%%%%%%%%%%%%%%%%%%%%%%%%%%%%%%%%%%%%%%%%%%%%%%%%%%%%%%%%%%%%%%%%%%%%%%%%%%%%%%%%%%%%%%%%%%%%%%%%%%%%%%%%%%%%%%%%
\begin{figure}[!p]

\begin{mathpar}
\inferrule*[right=(ASSGN)]
{\den{E}s = \some{n}}
{\confi{(s,h)}{x:=E} \pstep{}{} \confi{(s[x \mapsto n],h)}{\skp}}

\inferrule*[right=(READ)]
{\den{E}s = \some{n_1} \\
h(n_1) = \some{n_2}}
{\confi{(s,h)}{x:=[E]} \pstep{}{} \confi{(s[x \mapsto n_2],h)}{\skp}}

\inferrule*[right=(WRITE)]
{\den{E}s = \some{n_1} \\
h(n_1) \neq \none{} \\
\den{E'}s = \some{n_2}}
{\confi{(s,h)}{[E]:=E'} \pstep{}{} \confi{(s,h[n_1 \mapsto n_2])}{\skp}}

\inferrule*[right=(OUTPUT)]
{\den{E}\tau = \some{n}}
{\confi{\tau}{\out{E}} \pstep{}{[n]} \confi{\tau}{\skp}}

\inferrule*[right=(IF-TRUE)]
{\den{B}\tau = \some{\ttt{true}}}
{\confi{\tau}{\condfull{B}{C_1}{C_2}} \pstep{}{} \confi{\tau}{C_1}}

\inferrule*[right=(IF-FALSE)]
{\den{B}\tau = \some{\ttt{false}}}
{\confi{\tau}{\condfull{B}{C_1}{C_2}} \pstep{}{} \confi{\tau}{C_2}}

\inferrule*[right=(WHILE-TRUE)]
{\den{B}\tau = \some{\ttt{true}}}
{\confi{\tau}{\while{B}{C}} \pstep{}{} \confi{\tau}{\seq{C}{\while{B}{C}}}}

\inferrule*[right=(WHILE-FALSE)]
{\den{B}\tau = \some{\ttt{false}}}
{\confi{\tau}{\while{B}{C}} \pstep{}{} \confi{\tau}{\skp}}

\inferrule*[right=(SEQ)]
{\confi{\tau}{C_1} \pstep{}{o} \confi{\tau'}{C_1'}}
{\confi{\tau}{\seq{C_1}{C_2}} \pstep{}{o} \confi{\tau'}{\seq{C_1'}{C_2}}}

\inferrule*[right=(SKIP)]
{ }
{\confi{\tau}{\seq{\skp}{C}} \pstep{}{} \confi{\tau}{C}}

\inferrule*[right=(ZERO)]
{ }
{\confi{\tau}{C} \pstepn{}{0}{} \confi{\tau}{C}}

\inferrule*[right=(SUCC)]
{\confi{\tau}{C} \pstep{}{o} \confi{\tau'}{C'} \\
\confi{\tau'}{C'} \pstepn{}{n}{o'} \confi{\tau''}{C''} \\
n > 0}
{\confi{\tau}{C} \pstepn{}{n+1}{o ++ o'} \confi{\tau''}{C''}}
\end{mathpar}

\caption{Standard Operational Semantics}
\label{basesem}
\end{figure}
%%%%%%%%%%%%%%%%%%%%%%%%%%%%%%%%%%%%%%%%%%%%%%%%%%%%%%%%%%%%%%%%%%%%%%%%%%%%%%%%%%%%%%%%%%%%%%%%%%%%%%%%%%%%%%%%%%%%%%%

The operational semantics presented here is mixed-step and manipulates security labels directly. 
As mentioned above, we need to relate it to a more standard semantics that does not make
use of security labels. A standard, single-step semantics is defined in Figure~\ref{basesem}. 
This semantics operates on states without labels,
and it does not use continuation stacks. Given a state $\sigma$ with labels, we write $\bar{\sigma}$ to represent 
the same state with all labels erased from both the store and heap. We will also use $\tau$ to range over states 
without labels. Then the following two theorems hold:

\begin{thm}
Suppose $\pconfig{\sigma}{C}{[]} \psteps{\lo}{o} \pconfig{\sigma'}{\skp}{[]}$ in the instrumented
semantics. Then there exists $\tau$ such that $\confi{\bar{\sigma}}{C} \psteps{}{o} \confi{\tau}{\skp}$ 
in the standard semantics.
\end{thm}

\begin{thm}
Suppose $\confi{\bar{\sigma}}{C} \psteps{}{o} \confi{\tau}{\skp}$ in the standard semantics, and suppose
$\pconfig{\sigma}{C}{[]}$ never gets stuck when executed in the instrumented semantics. Then 
there exists $\sigma'$ such that $\bar{\sigma}' = \tau$ and
$\pconfig{\sigma}{C}{[]} \psteps{\lo}{o} \pconfig{\sigma'}{\skp}{[]}$ in the instrumented semantics.
\end{thm}

These theorems together guarantee that the two semantics produce identical observable behaviors (outputs)
on terminating executions, as long as the instrumented semantics does not get stuck. Our program logic will
of course guarantee that the instrumented semantics does not get stuck in any execution satisfying the precondition.

\begin{comment}
\begin{figure}
\begin{alltt}
// Assumption: Alice's calendar is a 64-element 
// array beginning at location A

1  i := 0;
2  while (i < 64) do
3      x := isFree(i);
4      if (x = true)
5          then 
6              output i
7          else 
8              skip;
9      i := i+1

10 isFree(i):
11     x := [A+i];
12     if (x = 0)
13         then
14             y := true;
15         else
16             y := false;
17     return y
\end{alltt}
\caption{Example: Alice's Private Calendar}
\label{calexample}
\end{figure}
\end{comment}




\begin{comment}
%%%%%%%%%%%%%%%%%%%%%%%%%%%%%%%%%%%%%%%%%%%%%%%%%%%%%%%%%%%%%%%%%%%%%%%%%%%%%%%%%%%%%%%%%%%%%%%%%%%%%%%%%%%%%%%%%%%%%%%
\begin{figure*}
\begin{mathpar}
\inferrule*[right=(OUTPUT-L)]
{\den{E}s = (n,\lo)}
{\pconfig{(s,\sigma,r)}{\out{E}}{K} \lstep{[n]} \pconfig{(s,\sigma,r)}{\skp}{K}}

\inferrule*[right=(PRIM-L)]
{(\bar{\sigma},s) \in \den{c}}
{\pconfig{\sigma}{\prim{c}}{K} \lstep{} \pconfig{\addlopols{\sigma}{\ttt{WS}(c)}{\ttt{RS}(c)}{s}}{\skp}{K}}

\inferrule*[right=(IF-LO-TRUE-L)]
{\den{B}\sigma = \ttt{Some }(\ttt{true},\lo)}
{\pconfig{\sigma}{\condfull{B}{C_1}{C_2}}{K} \lstep{} \pconfig{\sigma}{C_1}{K}}

\inferrule*[right=(IF-LO-FALSE-L)]
{\den{B}\sigma = \ttt{Some }(\ttt{false},\lo)}
{\pconfig{\sigma}{\condfull{B}{C_1}{C_2}}{K} \lstep{} \pconfig{\sigma}{C_2}{K}}

\inferrule*[right=(IF-HI-HALT-L)]
{\den{B}\sigma = \ttt{Some }(\_,\hi) \\
\hsafe{\pconfig{\sigma}{\condfull{B}{C_1}{C_2}}{[]}} \\
\pconfig{\sigma}{\condfull{B}{C_1}{C_2}}{[]} \hsteps \pconfig{\sigma'}{\skp}{[]}}
{\pconfig{\sigma}{\condfull{B}{C_1}{C_2}}{K} \lstep{} \pconfig{\sigma'}{\skp}{K}}

\inferrule*[right=(IF-HI-DVG-L)]
{\den{B}\sigma = \ttt{Some }(\_,\hi) \\
\hsafe{\pconfig{\sigma}{\condfull{B}{C_1}{C_2}}{[]}} \\
\forall \sigma' \such \pconfig{\sigma}{\condfull{B}{C_1}{C_2}}{[]} \hsteps\!\!\!\!\!\!\!\!\!\!\not \quad\;\; \pconfig{\sigma'}{\skp}{[]}}
{\pconfig{\sigma}{\condfull{B}{C_1}{C_2}}{K} \lstep{} \pconfig{\sigma}{\condfull{B}{C_1}{C_2}}{K}}

\inferrule*[right=(WHILE-LO-TRUE-L)]
{\den{B}\sigma = \ttt{Some }(\ttt{true},\lo)}
{\pconfig{\sigma}{\while{B}{C}}{K} \lstep{} \pconfig{\sigma}{C}{(\while{B}{C}) :: K}}

\inferrule*[right=(WHILE-LO-FALSE-L)]
{\den{B}\sigma = \ttt{Some }(\ttt{false},\lo)}
{\pconfig{\sigma}{\while{B}{C}}{K} \lstep{} \pconfig{\sigma}{\skp}{K}}

\inferrule*[right=(WHILE-HI-HALT-L)]
{\den{B}\sigma = \ttt{Some }(\_,\hi) \\
\hsafe{\pconfig{\sigma}{\while{B}{C}}{[]}} \\
\pconfig{\sigma}{\while{B}{C}}{[]} \hsteps \pconfig{\sigma'}{\skp}{[]}}
{\pconfig{\sigma}{\while{B}{C}}{K} \lstep{} \pconfig{\sigma'}{\skp}{K}}

\inferrule*[right=(WHILE-HI-DVG-L)]
{\den{B}\sigma = \ttt{Some }(\_,\hi) \\
\hsafe{\pconfig{\sigma}{\while{B}{C}}{[]}} \\
\forall \sigma' \such \pconfig{\sigma}{\while{B}{C}}{[]} \hsteps\!\!\!\!\!\!\!\!\!\!\not \quad\;\; \pconfig{\sigma'}{\skp}{[]}}
{\pconfig{\sigma}{\while{B}{C}}{K} \lstep{} \pconfig{\sigma}{\while{B}{C}}{K}}

\inferrule*[right=(SEQ-L)]
{ }
{\pconfig{\sigma}{\seq{C_1}{C_2}}{K} \lstep{o} \pconfig{\sigma}{C_1}{C_2 :: K}}

\inferrule*[right=(SKIP-L)]
{ }
{\pconfig{\sigma}{\skp}{C::K} \lstep{} \pconfig{\sigma}{C}{K}}
\end{mathpar}

\caption{Low Operational Semantics}
\label{lsem}
\end{figure*}
%%%%%%%%%%%%%%%%%%%%%%%%%%%%%%%%%%%%%%%%%%%%%%%%%%%%%%%%%%%%%%%%%%%%%%%%%%%%%%%%%%%%%%%%%%%%%%%%%%%%%%%%%%%%%%%%%%%%%%%

The following theorem shows that both our high and low instrumented semantics are valid refinements of our base
machine. If the high or low machine can transform one configuration to another in zero or more steps, then so can 
the base machine; additionally, the base machine will output the same thing as the high/low machines. The proof of this theorem 
is completely straightforward; details can be found in the extended technical report~\cite{ddifctr}. Note that the case 
for primitive commands rests on the fact that erasing all policies from $\addpols{\_}{\_}{\_}{s}$ gives back the 
original base state $s$.

\vspace{4mm}
\begin{thm}[Valid instrumented semantics]
\label{bsemvalid}
\begin{align*}
& \dt \, \forall \sigma,C,K,\sigma',C',K' \such \\
& \qquad \pconfig{\sigma}{C}{K} \hsteps \pconfig{\sigma'}{C'}{K'} \Longrightarrow \pconfig{\bar{\sigma}}{C}{K} \psteps{}{[]} \pconfig{\bar{\sigma}'}{C'}{K'} \\
& \dt \, \forall \sigma,C,K,\sigma',C',K',o \such \\
& \qquad \pconfig{\sigma}{C}{K} \lsteps{o} \pconfig{\sigma'}{C'}{K'} \Longrightarrow \pconfig{\bar{\sigma}}{C}{K} \psteps{}{o} \pconfig{\bar{\sigma}'}{C'}{K'} 
\end{align*}
\end{thm}
\end{comment}

