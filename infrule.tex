\newcommand\InfTagFont\textsc %[1]{#1}
\newcommand\EqnTagFont\textsc %[1]{#1}

%% a few macros to simplify defining commands with optional arguments

\makeatletter
\def\@withopt#1[#2]{#1{#2}}
\newcommand\optarg[2]{\@ifnextchar[{\@withopt{#2}}{\@withopt{#2}[#1]}}
\makeatother
\newcommand\compose[3]{#1{#2{#3}}}

%% Inference rules based on proof.sty 

% I don't remember who wrote these macros, but their design leaves a lot to be
% desired. First off, what might be the purpose of this tabular environment,
% which contains exactly one line and exactly one column? Second, setting the
% argument locally in math mode and enclosing the whole thing in a tabular
% prevents one from using these macros in an external array environment and
% placing formatting punctuation in the arguments (e.g. to align the judgments
% or tags of several axioms vertically).
% But it's probably too late to fix it for the wif paper, and having this code
% shared means it will only be getting harder.
% --val

\newcommand{\Axiom}[2][]{
       \begin{tabular}{@{}c@{}}
           \ifthenelse{\equal{#1}{}}{$#2$}{$#2 \;\;\; \InfTagFont{#1}$}
       \end{tabular}}

\newcommand{\Infer}[3][]{
       \begin{tabular}{@{}c@{}} \infer[\InfTagFont{#1}]{#2}{#3} \end{tabular}}
    
%\newcommand\EInferEqn[2]{
%  \begin{equation}%
%  \ifthenelse{\equal{#1}{}}{}{\tag{\EqnTagFont{#1}}}%
%  %% \label{eq:#1}%
%  \begin{array}{@{}c@{}}#2\end{array}%
%  \end{equation}}

\newlength\infrulewidth\setlength\infrulewidth{.855\columnwidth}
\newlength\infrulenamewidth\setlength\infrulenamewidth{.195\columnwidth}
\newlength\infrulevskip\setlength\infrulevskip{1ex}

\newcommand\EInferEqn[2]{
 \par\vspace\infrulevskip\noindent%
 \begin{tabular}{@{}p{\infrulewidth}@{}p{\infrulenamewidth}@{}}%
 \hfil$\begin{array}{@{}c@{}}#2\end{array}$\hfill &%
 \ifthenelse{\equal{#1}{}}{}{\hfill\EqnTagFont{(#1)}}\\[\infrulevskip]
%% \label{eq:#1}%
 \end{tabular}\par\medskip}

\newcommand\EAxiomFull[3]{\EInferEqn{#1}{#2#3}}
%% EAxiom is like EAxiomFull but has the parameter template [#1]#2[#3]
\newcommand\EAxiom{\optarg{}{\compose{\compose{\optarg{}}}\EAxiomFull}}

\newcommand\InferAligned[2]{\raisebox{-1.7ex}{\infer{#1}{#2}}}

\newcommand\EInferFull[4]{\EInferEqn{#1}{\InferAligned{#2}{#3}#4}}
%% EInfer is like EInferFull but with the template [#1]#2#3[#4]
\newcommand\EInfer{\optarg{}{\compose{\compose{\compose{\optarg{}}}}\EInferFull}}

%% Inference rules

\newcommand\RuleSec[2]{\noindent\fbox{\ensuremath{#1}}}
\newcommand\RuleSSec[2]{\noindent\fbox{\ensuremath{#1}}}

\newcommand\Eqn[3][]{%
  \begin{equation}%
  \ifthenelse{\equal{#1}{}}{}{\tag{\EqnTagFont{#1}}}%
  \label{eq:#2}%
  #3%
  \end{equation}}

\newcommand\Eqnu[3][]{\[#3\]}

\newcommand\Eq[2]{%
  \begin{equation*}%
  \label{eq:#1}
  #2%
  \end{equation*}}

\newcommand\Rulex[3]{%
  \frac
  {\begin{array}{@{\;}#1@{\;}}#2\end{array}}
  {\begin{array}{@{\;}#1@{\;}}#3\end{array}}
  }
\newcommand\Rule{\Rulex l}
\newcommand\Rulec{\Rulex c}

%\newcommand\Axiom[1]{%
%  \begin{array}{@{}l@{}}%
%    #1\\
%  \end{array}}

%{\theorembodyfont{\raggedright}     %% I commented this out for my
%  \newtheorem{lemma}{Lemma}         %% paper -- conflicting with 
%  \newtheorem{theorem}{Theorem}     %% the {theorem} package?? ---nadeem
%  }

\newif\iflongproofs\longproofstrue
\newif\iftabularprop\tabularpropfalse

\newenvironment{proofcases}
{\begin{list}{}{
      \renewcommand\makelabel[1]{\textbf{##1:} }
      \setlength\labelsep{0pt}
      \setlength\labelwidth{0pt}
      \setlength\leftmargin{.67em}
      \setlength\itemindent{0pt}
      }}
  {\end{list}}

\newcommand\PrFwCase{\ensuremath{(\Longrightarrow)}}
\newcommand\PrBkCase{\ensuremath{(\Longleftarrow)}}

\newenvironment{PrITE}{
  \iftabularprop
    \newcommand\PrIF{If&}
    \newcommand\PrSEP{\\&}
    \newcommand\PrAND{\\and&}
    \newcommand\PrTHEN{\\then&}
    \newcommand\PrWHERE{\\\emph{where}&}
    \begin{tabular}{@{}r@{\ }l@{}}
  \else
  \newcommand\PrIF{\textbf{If} }
  \newcommand\PrIFF{ \textbf{if and only if} }
  \newcommand\PrSEP{\textbf{,} }
  \newcommand\PrAND{\textbf{, and} }
  \newcommand\PrTHEN{\textbf{\ then} }
  \newcommand\PrWHERE{\textbf{\ \emph{where}} }
  \fi}{%
  \iftabularprop\end{tabular}%
  \else%
  \fi}
  
\newcommand\QED{\hfill$\Box$}

%% These expect these arguments:
%%   1) A textual name for the property
%%   2) An identifier naming the property, for cross-referencing
%%   3) A statement of the property
%%   4) A claim about how it can be proved 
%%        ("by induction on..", "by inspection")
%%   5) A summary of the proof, indicating key points
%%   6) The complete proof

\newcommand\Prop[7]{
  \begin{#1}[#2]\label{prop:#3} #4 \end{#1}}

\newcommand\PropPf[7]{
  \begin{#1}[#2]\label{prop:#3} #4 \end{#1}
  \textbf{Proof:} #5
  \iflongproofs #7
  \else #6
  \fi}

\newcommand\Lemma{\Prop{lemma}}
\newcommand\Theorem{\Prop{theorem}}

\newcommand\LemmaPf{\PropPf{lemma}}
\newcommand\TheoremPf{\PropPf{theorem}}


%%% Local Variables: 
%%% mode: latex
%%% TeX-master: "main"
%%% End: 
