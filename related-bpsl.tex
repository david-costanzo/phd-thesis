\label{related-chapter}

\section{Locality in Separation Logic}
\label{related-bpsl}

The definition of locality (or local action), which enables the frame
rule, plays a critical role in Separation
Logic~\cite{io00,reynolds02,yang:fossacs02}. Almost all versions of 
Separation Logic~--- including their
concurrent~\cite{brookes:concur04,ohearn:concur04,coy07},
higher-order~\cite{birkedal05}, and relational~\cite{yang07} variants,
as well as mechanized implementation (e.g., \cite{appel07})~--- have always used
the same locality definition that matches the well-known Safety and
Termination Monotonicity properties and the Frame
Property~\cite{yang:fossacs02}. 

In Chapter~\ref{bpsl-chapter}, we argued a case for strengthening the definition of locality 
to enforce \emph{behavior preservation}. This means that the behavior of a
program when executed on a small state is identical to the behavior when
executed on a larger state~--- put another way, excess, unused state cannot have
any effect on program behavior. We showed that this change can be made to have
no effect on the usage of Separation Logic, and we gave multiple examples of
how it simplifies reasoning about metatheoretical properties. 

\paragraph{Determinism Constancy}
One related work that calls for comparison is the property of
``Determinism Constancy'' presented by Raza and Gardner~\cite{rg09}, which
is also a strengthening of locality. While they use a slightly different notion 
of action than we do, it can be shown that Determinism Constancy, when translated
into our context (and ignoring divergence behaviors), is logically equivalent to:
\begin{align*}
\relate{\den{C}}{\sigma_0}{\sigma_0'} \land \sigma_0' \# \sigma_1 \Longrightarrow 
  \sigma_0 \# \sigma_1 \land \relate{\den{C}}{(\sigma_0 \dt \sigma_1)}{(\sigma_0' \dt \sigma_1)}
\end{align*}
\noindent{}For comparison, we repeat our Forwards Frame Property here:
\begin{align*}
\relate{\den{C}}{\sigma_0}{\sigma_0'} \land \sigma_0 \# \sigma_1 \Longrightarrow 
  \sigma_0' \# \sigma_1 \land \relate{\den{C}}{(\sigma_0 \dt \sigma_1)}{(\sigma_0' \dt \sigma_1)}
\end{align*}
While our strengthening of locality prevents programs from increasing state during
execution, Determinism Constancy prevents programs from \emph{decreasing} state.
The authors use Determinism Constancy to prove the same property regarding footprints
that we proved in Section~\ref{ssec:footprint}. Note that, while behavior preservation
does not imply Determinism Constancy, our concrete logic of Section~\ref{concrete} does
have the property since it never decreases state (we chose to have the \freee{} command put the 
deallocated cell back onto the free list, rather than get rid of it entirely).

While Determinism Constancy is strong enough to prove the footprint property, it does
not provide behavior preservation~--- an execution on a small state can still become
invalid on a larger state. Thus it will not, for example, help in resolving the dilemma 
of growing relations in the data refinement theory of~\cite{filipovic10}. Due to the lack of behavior 
preservation, we do not expect the property to have a significant impact on the metatheory
as a whole. Note, however, that there does not seem to be any harm in using \emph{both}
behavior preservation and Determinism Constancy. The two properties together enforce that
the area of memory accessible to a program be constant throughout execution.

\paragraph{Module Reasoning}
Besides our discussion of data refinement in
Section~\ref{ssec:datarefinement}, there has been some previous work
on reasoning about modules and their implementations within
the context of Separation Logic. In~\cite{oyr04},
a ``Hypothetical Frame Rule'' is used to allow modular reasoning when
a module's implementation is hidden from the rest of the
code. In~\cite{birkedal05}, a higher-order frame rule is used to allow
reasoning in a higher-order language with hidden module or function
code. However, neither of these works discuss relational reasoning
between different modules. We are not aware of any relational logic
for reasoning about modules.

\ifextended
Section~\ref{metatheory-security} discussed how behavior 
preservation is fundamentally important for combining local
reasoning with security verification; one interesting area
for future work would be to formalize this relationship
in the context of mCertiKOS. Currently, mCertiKOS uses a
single monolithic datatype for abstract state. In the future,
the framework could be altered to instead divide abstract
state into minimal composable pieces. This would yield clea
primitive specifications that only operate over the portion
of abstract state needed by the primitive (e.g., the \ttt{get\_quota}
specification would only take the \ttt{container} portion of abstract 
state as input). Behavior preservation would then need to be
explicitly enforced in order to \emph{soundly} and \emph{securely} combine 
all of these ``small'' specifications into a single, system-wide guarantee.

\begin{comment}
Another direction for future work would be to define a
behavior-preserving version of Concurrent Separation Logic that has
the same inference rules as standard CSL. The commands that acquire
and release locks should be able to be expressed in a
behavior-preserving fashion by including both local and shared state
in the underlying state model. A lock acquire will move memory from
the shared state into the local state, while a lock release will move
it from local into shared. Neither command requires an increase in
total state. The model could get quite interesting if we allow threads
to allocate memory. One possible way to implement this might be to
assign a separate free list to each thread. Another way might be to
use a single free list, and, at any point in execution, we consider
the free list to be owned by the currently-executing thread.
\end{comment}
\fi
